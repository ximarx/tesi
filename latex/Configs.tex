
\nocite{*}

\setcounter{secnumdepth}{3}
\setcounter{tocdepth}{2}

\frenchspacing % forza LaTeX ad una spaziatura non inglese
\hoffset + 0.5cm

\renewcommand{\sectionmark}[1]{%
\markboth{\MakeUppercase{%
\thesection. \ #1}}{}}


\rhead{\thesection \sectionmark}
\lfoot{}%\emph{Francesco Capozzo}}
\cfoot{\thepage}
\rfoot{}
\renewcommand{\headrulewidth}{0.4pt}
\renewcommand{\footrulewidth}{0.4pt}


\renewcommand{\contentsname}{Sommario}
\renewcommand{\listfigurename}{List of Figures}
\renewcommand{\listtablename}{List of Tables}
\renewcommand{\bibname}{Bibliografia}
\renewcommand{\indexname}{Indice}
\renewcommand{\figurename}{Figura}
\renewcommand{\tablename}{Tavola}
\renewcommand{\partname}{Parte}
\renewcommand{\chaptername}{Capitolo}
\renewcommand{\appendixname}{Appendice}
\renewcommand{\today}{\ifcase\month\or
  Gennaio\or Febbraio\or Marzo\or Aprile\or Maggio\or Giugno\or
  Luglio\or Agosto\or Settembre\or Ottobre\or Novembre\or Dicembre\fi
  \space\number\day, \number\year}
  
  
%riduce lo split delle footnote
\interfootnotelinepenalty=10000


\newtheorem{thm}{Teorema}

\theoremstyle{definition}
\newtheorem{defn}{Definizione}


\floatstyle{boxed}
%\newfloat{program}{thp}{lop}[chapter]
\newfloat{program}{tbphH}{lop}[chapter]
\floatname{program}{Codice}

\newcommand{\codefrom}[2][CLIPS]
{
\begin{program}[p]
    \lstinputlisting[language=#1]{#2}
    \caption{#2}
    \label{#2}
\end{program}
}

%\newenvironment{program}{\begin{boxedverbatim}}{\end{boxedverbatim}}


% Vincoli

% Initializing the counters and define a custom label
\newcommand{\vincoliinit}{
    % Create a new counter for keeping track of the last number
    \newcounter{vincolicountbackup}
    % Create a new counter for the custom label
    \newcounter{vincolicount}
    % Redefine the command for the last counter so when it is called
    % it prints the number like this in a bold font: R<number>
    \renewcommand{\thevincolicount}{\textbf{Vincolo-\arabic{vincolicount}: }}
}

% Used to define the start of the requirements
\newcommand{\vincolistart}{
    % Indicate the start of a new list and tell it to use the redefined
    % command and corresponding counter for every item
    \begin{list}{\thevincolicount}{\usecounter{vincolicount}}
    % Important part: set the value of the used counter to the
    % same value of the backup counter.
    \setcounter{vincolicount}{\value{vincolicountbackup}}
}

% Used to define the end of the requirements
\newcommand{\vincoliend}{
    % Important part: take the value of the used counter (after
    % being incremented by the requirement items) and store it
    % in the backup counter.
    \setcounter{vincolicountbackup}{\value{vincolicount}}
    % Mark the end of the list environment
    \end{list}
}


% COLORI

\definecolor{grigio-chiarissimo}{rgb}{0.9,0.9,0.9}


% grafici

\pgfplotsset{/pgf/number format/use comma,compat=newest}
