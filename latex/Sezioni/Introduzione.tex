\chapter*{Prefazione}
\addcontentsline{toc}{chapter}{Prefazione}
\chaptermark{Prefazione}
\rhead{}

Gli strumenti di sviluppo per sistemi esperti hanno vissuto un periodo di grande fermento durante gli anni '80~\cite{21aiproud}. L'introduzione di nuove tecniche algoritmiche, che permettessero la distribuzione e l'utilizzo dei sistemi prodotti su hardware di comune reperibilità, ha traghettato i sistemi basati su conoscenza da strumenti orientati alla ricerca anche verso ambiti con implicazioni più pratiche. Il grande passo avanti compiuto con la creazione di \emph{environment} in linguaggi diffusi ed efficienti come il C ha offerto la possibilità di adottare queste tecnologie anche in situazioni dove le risorse disponibili, sia economiche che computazionali, risultavano relativamente limitate. Il nuovo \emph{boom} tecnologico catalizzato dall'avvento di \emph{Internet} sta portando ad un cambiamento radicare nel modo di concepire il software ed i meccanismi di distribuzione. Il fenomeno sta portando alla luce molti dei limiti dei prodotti concepiti e realizzati prima dell'avvento del \emph{web}. Valori come le prestazioni e l'efficienza nella gestione delle risorse, pur rimanendo fattori importanti, sono sempre più spesso messi in secondo piano davanti a criteri di scelta legati alla rapidità e la facilità di sviluppo. La proprietà di un sistema di poter essere adattato più facilmente e più velocemente al repentino cambiamento delle necessità  è diventato un elemento chiave in qualsiasi scelta di sviluppo software.

Lo scopo di questo lavoro di tesi, realizzato presso il laboratorio LACAM del dipartimento di Informatica dell'Università degli Studi di Bari, è quello proporre un \emph{expert system environment} che renda agevole l'attività di integrazione dei sistemi esperti anche nei contesti che richiedano l'utilizzo di tecnologie per la distribuzione dei servizi tramite il web.

Nel primo capitolo si fornirà una panoramica sui sistemi esperti, il loro ambito di utilizzo e il processo di sviluppo. Si descriveranno le modalità di rappresentazione della conoscenza e si introdurranno le principali tecniche coinvolte per la realizzazione dell'inferenza e del ragionamento basato su regole, focalizzandosi sulla descrizione del problema del confronto fra stato del sistema e base di conoscenza. Verrà quindi effettuata una panoramica sugli strumenti di supporto allo sviluppo dei sistemi esperti fornendone una classificazione in base alle funzionalità e caratteristiche.

Nel secondo capitolo si approfondiranno le problematiche relative alla realizzazione di un \emph{environment} per sistemi esperti, analizzandone le componenti chiave e fornendo una descrizione delle tecniche utilizzate e delle capacità richieste al prototipo realizzato in questo lavoro di tesi.

Nel terzo capitolo  si affronteranno le tematiche relative alla progettazione ed all'implementazione del sistema secondo i vincoli evidenziati nel secondo capitolo. Verrà fornita una descrizione sommaria delle componenti che realizzano il sistema e degli algoritmi utilizzati dallo stesso. Si passerà quindi ad una sperimentazione del prototipo, necessaria per valutarne la correttezza e le prestazioni, offrendo un confronto con una soluzione di riferimento nel campo degli \emph{environment}, quale è CLIPS.