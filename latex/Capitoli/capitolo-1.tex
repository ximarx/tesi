\chapter{Environment per Sistemi Esperti}
Il numero di ambienti di sviluppo per sistemi esperti è cresciuto negli anni e con esso il numero e la tipologia di funzionalità offerte. Le prime shell semplici metodi di rappresentazione della conoscenza basata sull'uso delle regole, con il supporto di strumenti di inferenza piuttosto limitati. Con gli anni, la lista e la tipologia di funzionalità offerte si è arricchita introducendo una varietà di possibilità per la rappresentazione della conoscenza, strategie di ricerca, meccanismi specializzati di inferenza e strumenti di supporto allo sviluppo dei sistemi.

L'evoluzione di questi sistemi è stata motivata con il tempo e la quantità di sforzo richiesto per la costruzione di sistemi esperti usando linguaggi tradizionali e sistemi strettamente vincolati all'uso della regole per la rappresentazione della conoscenza.

L'obiettivo inseguito con l'evoluzione degli ambienti di sviluppo era quello di ridurre i tempi ed i costi di sviluppo dei sistemi esperti. Inoltre l'uso di questi strumenti ha garantito:
\begin{itemize}
	\item un miglioramento generale della qualità e l'affidabilità dei sistemi prodotti
	\item di astrarre l'attività di sviluppo del sistema esperto da quelle relative allo sviluppo dell'ambiente di base
	\item di permette agli ingegneri della conoscenza di focalizzarsi sulla modellazione degli elementi del dominio del sistema esperto
	\item di utilizzare strumenti rapidi per l'acquisizione e la modifica della conoscenza dei sistemi esperti
\end{itemize}

I miglioramenti hanno in questo modo incrementato le possibilità di successo nello sviluppo di sistemi esperti. Come ulteriore effetto, il ridursi della complessità generale di sviluppo ha consentito di gestire e produrre soluzioni di sempre maggiore complessità.

\section{Evoluzione degli strumenti di sviluppo}
I primi strumenti per lo sviluppo di sistemi esperti erano derivati da noti sistemi esperti. A titolo d'esempio, uno dei primi tool, EMYCIN, era essenzialmente una versione generalizzata del S.E. per la diagnosi medica MYCIN al quale erano state rimosse conoscenza di dominio. In sostituzione vennero integrate funzionalità per la gestione della base di conoscenza e per consentire la consultazione e la modifica della stessa all'utente, consentendo la creazione di una grande varietà di sistemi esperti basati sulla stessa logica di ragionamento del sistema originale, ma applicata a differenti ambiti di dominio. Quest'approccio, sebbene comodo di primo impatto, imponeva numerose tipologie di vincoli allo sviluppatore. Primo fra tutte l'impossibilità di ampliare o modificare le dinamiche di inferenza originali. Sistemi con logiche di sviluppo analoghe a quelle di EMYCIN prendono il nome di \emph{Expert System Shell}.



