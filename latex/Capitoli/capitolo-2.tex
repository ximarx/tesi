
\chapter{Analisi del sistema}

Lo scopo di questo capitolo è quello di fornire un'analisi del sistema da proporre. Verrà specificato il contesto nel quale l'artefatto dovrà operare, mettendo in evidenzia, in seguito ad una analisi preliminare, l'insieme dei requisiti che dovranno risultare soddisfatti. Per la struttura di questo capitolo verranno utilizzate come riferimenti le specifiche suggerite in \cite{ieee830-1998}.

\section{Scopo del sistema}

Il progetto \emph{MyCLIPS} si prefigge lo scopo di sviluppare un software di tipo \emph{Expert System Environment}, mediante la realizzazione di un ambiente multi-paradigmatico che supporti programmazione procedurale e basata su regole, un regime di controllo di tipo \emph{forward chaining} e che fornisca soluzioni per la distribuzione dei servizi tramite architettura \emph{client-server}.

L'artefatto derivante dall'implementazione del progetto \emph{MyCLIPS} non sarà un prodotto con fini commerciali, ma piuttosto uno strumento di supporto alla ricerca nell'ambito degli \emph{environment per sistemi esperti} e dell'integrazione degli stessi con tecnologie orientate al web.


\section{Definizioni, acronimi, abbreviazioni}

\begin{itemize}
	\item \emph{Ambiente di sviluppo integrato}:
	\item \emph{Analisi dei requisiti}:
	\item \emph{API}:
	\item \emph{Best Practice}:
	\item \emph{Bug}:
	\item \emph{C}:	
	\item \emph{Client}:
	\item \emph{Casi d'uso}:
	\item \emph{Codice sorgente}:
	\item \emph{CPU}:
	\item \emph{Design Pattern}:
	\item \emph{Eclipse}:
	\item \emph{Expert System Environment}:
	\item \emph{Forward Chaining}:
	\item \emph{GUI}:
	\item \emph{IDE}:
	\item \emph{Incapsulamento}:
	\item \emph{Incrementale}:
	\item \emph{Java}:
	\item \emph{Javadoc}:
	\item \emph{Multipiattaforma}:
	\item \emph{MyCLIPS}:
	\item \emph{Open source}:
	\item \emph{Overhead}:
	\item \emph{Package}:
	\item \emph{Pattern Matching}:
	\item \emph{Plugin}:
	\item \emph{PyDev}:
	\item \emph{Python}:
	\item \emph{RETE}:
	\item \emph{Strategia di risoluzione dei conflitti}:
	\item \emph{Rule-Based Systems}:
	\item \emph{Server}:
	\item \emph{Sistema Operativo}:
	\item \emph{Skeleton}:
	\item \emph{Standalone}:
	\item \emph{Stub}:
	\item \emph{Thread}:
	\item \emph{UML}:
	\item \emph{Unificazione}:
	\item \emph{Use Case}:
	\item \emph{XML}:
	\item \emph{XMLRPC}:
\end{itemize}

\section{Riferimenti teorici}

In questa sezione si forniscono dei riferimenti a basi teoriche necessarie per la comprensione del problema che il progetto punta a risolvere.


\subsection{Rule-Based Systems}

Una descrizione generale dell'architettura del sistemi a regole. Richiamare brevemente il paragrafo Regole-di-Produzione.

\paragraph{Definizione di un formalismo per le regole}

Fornire esempi di regole, esempi in CLIPS magari

\paragraph{Componenti di un RBS}

[DIAGRAMMA delle componenti RBS]

Specifica delle componenti dettagliata, con richiami a quanto descritto nel capitolo 1 se c'e' gia.

\subparagraph{Motore inferenziale}

Richiamo veloce

\subparagraph{Working memory}

Richiamo veloce. Spiegare il concetto di fatti iniziali.

\subparagraph{Memoria delle regole}

Spiegazione veloce. Richiami a LHS, RHS e al concetto di Pattern

\subparagraph{Conflict Set}

Spiegare a cosa serve, introdurre le strategia di risoluzione

\subsection{Unificazione e Pattern Matching}

Spiegare il problema del Pattern Matching in maniera formale. Riferimento: \begin{verse}
http://lacam.di.uniba.it:8000/~ferilli/corsi/icse/confronto.pdf

http://lacam.di.uniba.it:8000/~ferilli/ufficiale/corsi/icse/confronto.pdf
\end{verse}

\subsection{Algoritmo RETE}

Riferimento a Forgy, descrizione originale e poi a tesi CMU per versione interpretata.

Riferimento a paragrafo in capitolo 1.

Magari riferimento a Rete II/UL

\subsection{Risoluzione dei conflitti}

Riferimento a strategie in capitolo 1, esempio clips e descrizione veloce delle sue strategie.

\subsection{Architetture per il calcolo distribuito}

Riferimento a Client-Server. Definizione generale dell'architettura

\section{Capacità}

Bla bla

\section{Stakeholders}

Bla bla

\section{Vincoli di sistema}

Bla bla

\subsection{Requisiti di interfaccia}

Bla bla

\subsection{Requisiti operativi}

Bla bla

\subsection{Requisiti prestazionali}

Bla bla

\subsection{Requisiti progettuali}

Bla bla

\subsection{Attributi di sistema}

\subsubsection{Affidabilità}

\subsubsection{Manutenibilità}

\subsubsection{Portabilità}

Bla bla

\subsection{Stabilità dei vincoli}

Bla bla

\section{Attori e Casi d'uso}

Bla bla

\subsection{Scenari dei casi d'uso}

Bla bla