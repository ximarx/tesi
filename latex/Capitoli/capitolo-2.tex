
\chapter{Analisi del sistema}

Lo scopo di questo capitolo è quello di fornire un'analisi del sistema da proporre. Verrà specificato il contesto nel quale l'artefatto dovrà operare, mettendo in evidenzia, in seguito ad una analisi preliminare, l'insieme dei requisiti che dovranno risultare soddisfatti. Per la struttura di questo capitolo verranno utilizzate, come riferimenti, le specifiche suggerite in \cite{ieee830-1998}.

\section{Scopo del sistema}

Il progetto \emph{MyCLIPS} si prefigge lo scopo di sviluppare un software di tipo \emph{Expert System Environment}, mediante la realizzazione di un ambiente multi-paradigmatico che supporti programmazione procedurale e basata su regole, un regime di controllo di tipo \emph{forward chaining} e che fornisca soluzioni per la distribuzione dei servizi tramite architettura \emph{client-server}.

L'artefatto derivante dall'implementazione del progetto \emph{MyCLIPS} non sarà un prodotto con fini commerciali, ma piuttosto uno strumento con fini didattici e di supporto alla ricerca nell'ambito degli \emph{environment per sistemi esperti} e dell'integrazione degli stessi con tecnologie orientate al Web.

\section{Definizioni, acronimi, abbreviazioni}

\begin{itemize}
	\item \emph{Ambiente di sviluppo integrato}:
	\item \emph{Analisi dei requisiti}:
	\item \emph{API}:
	\item \emph{Best Practice}:
	\item \emph{Bug}:
	\item \emph{C}:	
	\item \emph{Client}:
	\item \emph{Casi d'uso}:
	\item \emph{Codice sorgente}:
	\item \emph{CPU}:
	\item \emph{Design Pattern}:
	\item \emph{Eclipse}:
	\item \emph{Expert System Environment}:
	\item \emph{Forward Chaining}:
	\item \emph{GUI}:
	\item \emph{IDE}:
	\item \emph{Incapsulamento}:
	\item \emph{Incrementale}:
	\item \emph{Java}:
	\item \emph{Javadoc}:
	\item \emph{Listener}:
	\item \emph{Multipiattaforma}:
	\item \emph{MyCLIPS}:
	\item \emph{Multi-slot}:
	\item \emph{Open source}:
	\item \emph{Overhead}:
	\item \emph{Package}:
	\item \emph{Pattern Matching}:
	\item \emph{Plugin}:
	\item \emph{PyDev}:
	\item \emph{Python}:
	\item \emph{RETE}:
	\item \emph{Slot}:
	\item \emph{Strategia di risoluzione dei conflitti}:
	\item \emph{Rule-Based Systems}:
	\item \emph{Server}:
	\item \emph{Sistema Operativo}:
	\item \emph{Skeleton}:
	\item \emph{Standalone}:
	\item \emph{Stub}:
	\item \emph{Thread}:
	\item \emph{UML}:
	\item \emph{Unificazione}:
	\item \emph{Use Case}:
	\item \emph{XML}:
	\item \emph{XMLRPC}:
\end{itemize}

%%%%% ATTIVARE PER COMPILAZIONE RAPPORTO-TECNICO %%%%%
% Include:
% 	- Sezione: Riferimenti teorici
%		- RBS
%		- Unificazioni
%		- CRS
%		- CLIPS
%		- Architetture per sistemi distribuiti
%\section{Riferimenti teorici}

In questa sezione si forniscono dei riferimenti a basi teoriche necessarie per la comprensione del problema che il progetto punta a risolvere.


\subsection{Rule-Based Systems}

I sistemi di produzione sono un tipo di \emph{Pattern Directed Inference System}. Rappresentano un modello computazionale con regime di controllo guidato da \emph{pattern}. 

Il processo decisionale viene guidato attraverso l'utilizzo di regole: strutture costituite da coppie \emph{antecedente-conseguente}.

La parte \emph{antecedente}, definita anche \emph{LHS} o \emph{condizione}, contiene la rappresentazione di uno schema di caratteristiche che devono corrispondere nella descrizione dello stato corrente del sistema. Nel caso questi schemi (\emph{pattern}) risulteranno verificati, la regola verrà definita \emph{attivabile}. L'attivazione della regola non è immediata, ma gestita in un flusso di controllo ben prestabilito. Con il termine \emph{attivazione} verrà definita la coppia formata da una \emph{regola attivabile} e l'insieme di caratteristiche presenti nella descrizione dello stato che hanno reso la regola \emph{attivabile}.

La parte \emph{conseguente}, definita anche \emph{RHS} o \emph{azione}, rappresenta l'azione (o l'insieme di azioni) che dovranno essere eseguite nel momento in cui una regola \emph{attivabile} verrà effettivamente attivata. Solitamente il gruppo conseguente consiste in una serie di modifiche da apportare alla descrizione dello stato del sistema, ma questa generalizzazione non rappresenta una restrizione.

Il funzionamento di questa classe di sistemi può essere espresso come una sequenza di cicli \emph{recognize-act}:
\begin{description}
	\item[recognize:] viene valutata l'applicabilità di ogni regola in relazione allo stato corrente del sistema. La valutazione viene effettuata confrontando lo stato del sistema con la porzione \emph{LHS} di ogni regola.
	\item[selezione:] vengono applicati dei principi di selezione per individuare una o più regole fra quelle attivabili. L'insieme di principi utilizzati viene definito con il nome di \emph{strategia di risoluzione di conflitti}.
	\item[act:] la regola selezionata viene attivata: l'elenco di azioni presenti nella porzione \emph{RHS} vengono eseguite.
\end{description}

\'E evidente come il ciclo sia strutturato in modo che il momento di esame dei dati sia nettamente separato dal momento di elaborazione o modifica degli stessi. Inoltre l'unica modalità per lo scambio di informazioni fra le regole prevista, avviene attraverso la modifica stessa dello stato del sistema presente nella \emph{working memory}. Questa caratteristica differenzia ulteriormente questa classe di sistemi da quelli con approccio procedurale: se l'approccio basato su regole di produzione prevede che la stessa descrizione dello stato sia condivisa fra tutte le regole e che l'accesso allo stesso sia garantito in egual misura, il modello procedurale prevede che le singole procedure accedano ad un ristretto (e focalizzato) sottoinsieme dei dati presenti nello stato globale.

\subsubsection{Componenti di un RBS}


La struttura classica di un sistema basato su regole può essere schematizzata (\figurename~\ref{fig:architettura-rbs}) attraverso la specifica dei quattro componenti principali da cui è composto.

\begin{figure}[h]
\centering
\includegraphics[viewport=28 667 361 824]{Immagini/Capitolo2/Architettura-RBS.pdf}
\caption{Architettura generalizzata di un Rule-Based System}\label{fig:architettura-rbs}
\end{figure}

\subparagraph{Memoria di lavoro}
La memoria di lavoro, anche chiamata \emph{working memory}, è il componente destinato ad accogliere la descrizione simbolica dello stato del sistema. La descrizione è composta da strutture simboliche note con il nome di \emph{fatti}. La forma con la quale i fatti vengono rappresentati dipende dal sistema spesso, ma formalismi comunemente utilizzati prevedono una rappresentazione tramite \emph{vettori di caratteristiche} o \emph{triple (oggetto, proprietà, valore)}.

La \emph{working memory} viene utilizzata dal motore inferenziale per il processo di verifica della possibilità di attivazione delle regole.

\subparagraph{Motore inferenziale}

Il \emph{motore inferenziale}, o \emph{inference engine}, rappresenta il nucleo centrale di ogni sistema a regole: è il componente con il compito di valutare la parte \emph{LHS} delle regole, identificare quelle attivabili compilando il \emph{conflict-set} e infine, in seguito ad un processo di selezione guidato dalla \emph{strategia di risoluzione dei conflitti}, eseguirne una.

Il processo alla base della valutazione delle regole applicabili prende il nome di \emph{pattern-matching}.

\subparagraph{Memoria delle regole}
La memoria delle regole, o \emph{rules-set}, è una struttura che conserva l'insieme di regole definite nel sistema. Viene consultata dal motore inferenziale per individuare le regole attivabili per uno stato del sistema.

\subparagraph{Conflict-Set}
Il \emph{conflict-set}, anche noto in determinati sistemi con il nome di \emph{agenda} o \emph{agenda delle attivazioni}, è un componente che conserva l'insieme di \emph{attivazioni} disponibili per lo stato corrente della \emph{working-memory}. L'insieme viene compilato dal \emph{motore inferenziale} al termine del passo \emph{recognize} che caratterizza questa classe di sistemi.

Il \emph{conflict-set} viene consultato per effettuare una selezione nell'insieme di attivazioni disponibili, utilizzando una \emph{strategia di risoluzione dei conflitti}, e individuare l'attivazione da eseguire nel passo \emph{act}.

\subsection{Unificazione e Pattern Matching}\label{par:pattern-matching}

Il problema del \emph{confronto di descrizioni simboliche} è uno dei problemi alla base dei sistemi basati su conoscenza. Una forma di questo genere di problemi è quello dell'\emph{unificazione} o \emph{matching}.

L'\emph{unificazione} rappresenta un procedimento con il quale vengono confrontate fra loro un insieme di due o più strutture simboliche per identificarne differenze o somiglianze.~\cite{ferilli200}

Dati due termini $s$ e $t$, il problema dell'unificazione consiste nel capire se esistono dei termini in grado di essere sostituiti alle variabili che compaiono in $s$ e $t$ affinché i termini ottenuti risultino identici.~\cite{ferilli200}

Per molte applicazioni pratiche, l'unificazione risulta un concetto troppo generale. Si fa quindi riferimento al \emph{Pattern Marching}, una variante dell'unificazione nella quale è possibile effettuare sostituzioni solo in una delle due strutture simboliche.~\cite{ferilli200}


\begin{defn}[Termini confrontabili]
Dati due termini, $s$ e $t$, si dice che \emph{esiste un matching} tra essi se esiste una sostituzione $\mu$ tale che:
\begin{equation}
\mu(s)=t \text{ oppure } \mu(t)=s
\end{equation}
In tal caso $\mu$ è definito \emph{matcher} di $s$ e $t$, mentre $\mu(s)$ (risp. $\mu(t)$) è definito \emph{matching} di $s$ e $t$.
\end{defn}

\subsection{Strategie di risoluzione dei conflitti}\label{par:strategies}

Come visto in precedenza, la fase \emph{recognize} ha lo scopo di valutare eventuali cambiamenti apportati allo stato del sistema identificando l'insieme di \emph{attivazioni} disponibili. L'elenco di \emph{attivazioni} verrà memorizzato nel \emph{conflict-set}. 

In una fase successiva, l'utilizzo di determinate euristiche provvederà a definire dei criteri con i quali discriminare l'insieme di attivazioni per identificarne un sotto-insieme\footnote{Solitamente composto di un solo elemento.}.

L'insieme di criteri con i quali discriminare l'insieme prende il nome di \emph{strategia di risoluzione dei conflitti}.

\subsection{CLIPS}\label{par:clips}

CLIPS (C Language Integrated Production System) è uno dei più conosciuti \emph{Expert System Environment}. Implementato in linguaggio C presso i laboratori NASA del Johnson Space Center, è attualmente uno dei sistemi più utilizzati in ambito accademico grazie alla sua reperibilità e gratuità.

CLIPS è un ambiente multi-paradigmatico che integra all'interno di un'architettura di base di tipo \emph{RBS}, elementi per la programmazione procedurale e orientata ad oggetti.

L'interazione con il sistema può avvenire attraverso l'utilizzo di una \emph{Shell}, che consente l'immissione di costrutti, l'esecuzione di query e chiamate a funzioni, oppure attraverso l'integrazione del sistema in applicazioni classiche, che ne sfruttino i servizi tramite apposite API.

Il sistema prevede dei meccanismi di estensione tramite la compilazione congiunta di funzioni in linguaggio C, le quali verranno rese disponibili nel sistema in maniera analoga a quelle integrate in origine.

\subsubsection{Presentazione dei formalismi di specifica}\label{par:clips-formalism}

Il linguaggio di specifica utilizzato da CLIPS, prevede una sintassi simile a quella utilizzata da LISP. La definizione dei costrutti (\emph{regole}, \emph{template}, \emph{funzioni}, \emph{fatti}, $\dots$) avviene utilizzando un linguaggio chiamato CLIPS, analogamente al sistema. Un'estensione del linguaggio fornito originariamente, che prende il come di \emph{COOL}, è stata aggiunta nelle versioni successive per integrare il supporto il paradigma \emph{object-oriented}.



\subsection{Architetture per sistemi distribuiti}

Un sistema distribuito consiste di una collezione di componenti, distribuiti su diversi terminali (anche chiamati \emph{host}) connessi fra loro attraverso una rete. Questi componenti interagiscono fra loro al fine di scambiare dati od accedere reciprocamente a servizi disponibili.~\cite{Mascolo:2002:MCM:770420.770423}

Ad alto livello, la realizzazione di un sistema distribuita è soggetta alla predisposizione di un sistema di comunicazione in grado di mettere in comunicazione i processi distribuiti fra i vari \emph{host}.

Sono disponibili diverse architetture per strutturare un sistema distribuito. Le più note prendono i nomi di \emph{Client-Server}, \emph{3-tier Architecture}, \emph{n-tier Architecture}, \emph{Peer-to-Peer}, solo per citarne alcune.

\subsubsection{Modello Client-Server}
Il modello di architettura \emph{Client-Server} si basa sul partizionamento delle attività, risorse e responsabilità del sistema in due classi di componenti: \emph{Server} e \emph{Client}.

I componenti possono essere distribuiti su diversi \emph{host}, così come possono essere collocati in processi separati residenti sulla stessa macchina in modo che non condividano direttamente risorse (\figurename~\ref{fig:client-server}). Lo scambio di informazioni avviene attraverso un protocollo stabilito e noto ad entrami i sistemi.

\begin{figure}[h]
\centering
\includegraphics[scale=0.7, viewport=6 582 546 838]{Immagini/Capitolo2/Client-Server.pdf}
\caption[Architettura \emph{Client-Server}]{Architettura \emph{Client-Server}: in \emph{(a)} è mostrata una distribuzione su \emph{host} multipli, in \emph{(b)} una distribuzione in processi multipli}\label{fig:client-server}
\end{figure}

L'architettura \emph{Client-Server} descrive un modello di collaborazione fra sistemi. Il \emph{server} offre servizi o funzionalità a uno o più \emph{client}, mentre il \emph{client} effettua una richiesta e, solitamente, provvede ad elaborare i dati ottenuti come risposta dal \emph{server} per presentarli all'utente o proseguire con ulteriori elaborazioni.

\subsubsection{Protocollo XML-RPC}\label{par:xmlrpc}

\emph{Remote Procedure Call}, o \emph{RPC}, è un meccanismo tramite il quale un'applicazione posizionata su una \emph{macchina} richiede l'utilizzo di servizi forniti da un'altra applicazione residente su una \emph{macchina} differente. Con il termine \emph{macchina} non viene unicamente inteso un \emph{host}, un'installazione fisicamente indipendente, ma anche un processo residente in una porzione di memoria indipendente e che rende l'accesso alle risorse del servizio impossibile se non tramite il meccanismo \emph{RPC}. 

Il meccanismo \emph{RPC} prevede che una prima applicazione possa inviare uno o più messaggi ad una applicazione differente per invocare delle procedure. L'applicazione ricevente ha la possibilità,  a sua volta, di inviare i risultati dell'elaborazione tramite un ulteriore messaggio verso applicazione richiedente.~\cite{MERRICK:2006:misc}

\emph{RPC} descrive un meccanismo generalizzato per l'invocazione di procedure remote, non una concreta implementazione di un protocollo o di tecnologie per lo scambio dei messaggi fra gli \emph{host}. Partendo dal concetto generale di \emph{RPC}, nel tempo sono state proposte differenti implementazioni basate su protocolli proprietari o codifiche dei messaggi differenti.~\cite{JAIRATH:2004:misc}~\cite{dcerpc}

\emph{XML-RPC} è una fra le implementazioni proposte per \emph{RPC}: si propone come un meccanismo di chiamata a procedura remote semplice e versatile. Il brevetto originale con il quale è stato proposto descrive sia il meccanismo di chiamata, che un sistema per l'implementazione del metodo.~\cite{MERRICK:2006:misc}

\begin{figure}[h]
\centering
\includegraphics[scale=0.5, viewport=0 0 646 440]{Immagini/Capitolo2/XMLRPC-patent.pdf}
\caption[Comunicazione tramite \emph{XML-RPC}]{Comunicazione tramite \emph{XML-RPC}: immagine presente nella richiesta originale di brevetto~\cite{MERRICK:2006:misc}.}\label{fig:xmlrpc-patent}
\end{figure}

\emph{XML-RPC} prevede lo scambio di richieste e risposte usando il formato \emph{XML}, attraverso meccanismi di trasporto multipli. Solitamente, con l'uso del termine \emph{XML-RPC} si è soliti però indicare il protocollo di scambio messaggi in \emph{XML} attraverso un meccanismo di trasporto basato sul protocollo \emph{HTTP}.

\begin{program}
\begin{verbatimtab}

POST /RPC2 HTTP/1.0
User-Agent: Frontier/5.1.2 (WinNT)
Host: betty.userland.com
Content-Type: text/xml
Content-length: 181


<?xml version="1.0"?>
<methodCall>
   <methodName>examples.getStateName</methodName>
   <params>
      <param>
         <value><i4>41</i4></value>
         </param>
      </params>
   </methodCall>
\end{verbatimtab}
\caption{Esempio di chiamata ad una procedura remota usando \emph{XML-RPC over HTTP}}\label{code:xmlrpc-request}
\end{program}

La codifica dei messaggi viene effettuata incapsulando i parametri di chiamata ed i valori di ritorno all'interno di appositi frammenti del documento \emph{XML}, i quali hanno il compito di attribuire una semantica alla stringa rappresentante il valore.

\begin{program}
\begin{verbatimtab}
HTTP/1.1 200 OK
Connection: close
Content-Length: 158
Content-Type: text/xml
Date: Fri, 17 Jul 1998 19:55:08 GMT
Server: UserLand Frontier/5.1.2-WinNT


<?xml version="1.0"?>
<methodResponse>
   <params>
      <param>
         <value><string>South Dakota</string></value>
         </param>
      </params>
   </methodResponse>
\end{verbatimtab}
\caption{Esempio di messaggi inviato a risposta di una chiamata a procedura remota usando \emph{XML-RPC over HTTP}}\label{code:xmlrpc-response}
\end{program}

I \emph{tag} che contengono il valore ne determinano il tipo. Nell'esempio fornito in Codice~\ref{code:xmlrpc-request}, il valore ''41'' (tipo \emph{integer}) viene incapsulato nel \emph{tag i4}. Il tag indica appunto che il tipo del valore, come da specifica, è un \emph{4 bytes signed integer}.

La specifica prevede la possibilità di utilizzare un set di tipi atomici con i quali rappresentare tutti i parametri di chiamata a procedura o i valori di ritorno (''integer'', ''string'', ''boolean'', ''double'', ''dateTime.iso8601'', ''base64'') e un gruppo di strutture composte (''struct'' per strutture analoghe a dizionari, ''array'' per vettori).~\cite{xmlrpcspec}


\section{Capacità}

Si riassumono quindi le funzionalità richieste al sistema.

\begin{enumerate}
	\item l'uso di un linguaggio di specifica che consenta:
		\begin{enumerate}
			\item la definizione di regole di produzione (\ref{par:clips-formalism})
			\item l'utilizzo di un paradigma procedurale (\ref{par:clips-formalism})
			\item la specifica di funzioni utente (\ref{par:clips-formalism})
			\item la specifica di moduli (\ref{par:clips-formalism})
			\item definizione di template di fatti (\ref{par:clips-formalism})
			\item la specifica della conoscenza fattuale tramite:
				\begin{enumerate}
					\item fatti in notazione Ordered-Fact (\ref{par:clips-formalism})
					\item fatti in notazione Template-Fact (\ref{par:clips-formalism})
				\end{enumerate}
		\end{enumerate}
	\item la possibilità di aggiungere strategie di risoluzione dei conflitti (\ref{par:strategies})
	\item la possibilità di integrare nuove funzioni di sistema (\ref{par:clips})
	\item compatibilità con i sistemi sviluppati per CLIPS (\ref{par:clips})
	\item l'utilizzo dell'algoritmo RETE per le operazioni di \emph{Pattern-Matching}~(\ref{par:pattern-matching})
	\item la possibilità di accedere alle funzionalità del sistema tramite una \emph{shell}~(\ref{par:clips})
	\item la possibilità di accedere alle funzionalità del sistema tramite protocollo XML-RPC~(\ref{par:xmlrpc})
\end{enumerate}


\section{Vincoli di sistema}
\vincoliinit
\subsection{Vincoli di interfaccia}

\'E prevista l'interazione fra utente e sistema attraverso l'utilizzo di una \emph{Shell} di comandi per permettere la specifica e l'esecuzione dei sistema esperto.  A soddisfacimento della capacità richiesta 7, è previsto l'utilizzo del sistema tramite protocollo XML-RPC che consenta agli utenti di integrare le funzionalità del sistema in applicazioni di terze parti. In base a queste premesse si esplicitano i seguenti vincoli d'interfaccia.
\vincolistart
	\item Hardware\\	
	L'utente si interfaccia al sistema inserendo appositi comandi tramite tastiera. Inoltre acquisisce i risultati dell'elaborazione dei comandi attraverso il monitor. La tastiera viene inoltre utilizzata per l'inserimento di dati, qualora necessario.
	
	\item Utente\\
	L'interazione con il sistema avviene attraverso un'interfaccia testuale composta da una linea di comando, nella quale l'utente esplicita i comandi e attende i risultati dell'elaborazione. Anche i risultati vengono presentati nella medesima interfaccia testuale.
	
	\item Software\\
	L'utilizzo del sistema attraverso l'integrazione in applicazioni di terze parti (tramite integrazione diretta o interfaccia XML-RPC) prevede l'utilizzo di applicazioni per la scrittura di software o ambienti di sviluppo integrati (IDE).
	
	\item Software\\
	Il sistema prevede la possibilità di utilizzare applicazioni di \emph{text-editing} per la creazione dei sistemi esperti.
	
\vincoliend

\subsection{Vincoli operativi}

\vincolistart
	\item Hardware\\
	Per l'utilizzo in modalità servizio XML-RPC:
	\begin{itemize}
		\item 1GB di memoria RAM;
		\item Interfaccia di rete.
	\end{itemize}
	\item Software\\
	\'E previsto l'utilizzo del sistema su piattaforma PC con sistemi operativi
	\begin{itemize}
		\item Microsoft Windows XP, Vista, 7
		\item Ubuntu 12.04
	\end{itemize}
\vincoliend

\subsection{Vincoli prestazionali}
\vincolistart
	\item le prestazioni complessive del sistema devono permettere il completamento dei test di una \emph{benchmark-suite} in tempo ragionevole.
\vincoliend
\subsection{Vincoli di progetto}
\vincolistart
	\item il sistema deve permettere l'esecuzione di artefatti realizzati tramite CLIPS (escludendo quelli che utilizzano le estensioni del linguaggio COOL), richiedendo soltanto modifiche di lieve entità.
	
	\item il sistema deve prevedere la possibilità di integrare nuove funzionalità
\vincoliend

\subsection{Stabilità dei vincoli}

\begin{longtable}{|c|c|c|p{7cm}|}
\caption{Matrice di stabilità dei vincoli}\label{tab:stabilita-vincoli}\\
\hline\hline
\rowcolor{grigio-chiarissimo} \textbf{\#} & \textbf{Stabile} & \textbf{Instabile} & \textbf{Motivazione} \\
\hline\hline
\endfirsthead
\hline\hline
\rowcolor{grigio-chiarissimo} \textbf{\#} & \textbf{Stabile} & \textbf{Instabile} & \textbf{Motivazione} \\
\hline\hline
\endhead

Vincolo-1 &  & $\blacklozenge$ & Nuove tecnologie potrebbero far migrare il sistema verso nuove forme di interazione \\ 
\hline 
Vincolo-2 & $\blacklozenge$ &  &  \\ 
\hline 
Vincolo-3 & $\blacklozenge$ & & \\ 
\hline 
Vincolo-4 & $\blacklozenge$ & & \\ 
\hline 
Vincolo-5 &  & $\blacklozenge$ & Lo sviluppo tecnologico potrebbe portare a richieste hardware differenti come conseguenza di un ampliamento delle richieste software \\ 
\hline 
Vincolo-6 & $\blacklozenge$ & & \\ 
\hline 
Vincolo-7 & & $\blacklozenge$ & Lo sviluppo del sistema potrebbe portare all'introduzione di vincoli prestazionali \\ 
\hline 
Vincolo-8 & & $\blacklozenge$ & Lo sviluppo del sistema potrebbe portare all'ampliamento della compatibilità o a modifiche del linguaggio che riducano la portabilità degli artefatti\\ 
\hline 
Vincolo-9 & $\blacklozenge$ & & \\ 
\hline

\end{longtable}


\section{Attori e Casi d'uso}

\subsection{Attori}

In seguito ad un'analisi preliminare, sono state individuate sette classi di attori principali (\figurename~\ref{fig:attori}).

\begin{figure}[h]
\centering
\includegraphics[width=0.9\textwidth]{Immagini/Capitolo2/UseCases/Attori.png}
\caption{Gerarchia degli attori}\label{fig:attori}
\end{figure}

\begin{description}
	\item[ES Consumer:] rappresenta un attore generalizzato in grado di utilizzare i servizi principali offerti dall'\emph{environment} per caricare, eseguire ed interagire con un artefatto precedentemente creato tramite i protocolli offerti dall'\emph{environment} stesso.
	
	\item[ES Developer:] specializzazione di \emph{ES Consumer}, servendosi del linguaggio di specifica fornito dall'\emph{environment} realizza artefatti eseguibili. Tramite l'utilizzo di apposite interfacce è anche in grado di testare l'artefatto prodotto analizzando rappresentazioni grafiche delle strutture interpretate e registri d'attività dell'esecuzione dell'artefatto.
	\item[Utente Umano:] specializzazione di \emph{ES Consumer}.
	
	\item[Standalone Application:] specializzazione di \emph{ES Consumer}. Utilizza i servizi offerti dall'\emph{environment} all'interno di applicazioni \emph{stand-alone}. Esegue artefatti e valuta i risultati dell'esecuzione analizzando eventi e lo stato finale del sistema.
	
	\item[Remote Application:] specializzazione di \emph{Standalone Application}. Esegue attività analoghe a quelle di \emph{Standalone Application}, ma utilizzando i servizi tramite appositi protocolli di comunicazione. Per supportare questi protocolli l'attore può creare, configurare o distruggere sessioni.
	
	\item[Environment Developer:] utilizzando la documentazione di sistema e le apposite \emph{API} fornite, estende le funzionalità dell'\emph{environment} aggiungendo nuove funzioni di sistema, strategie di risoluzione dei conflitti o \emph{event-listener}. Utilizzando un framework per l'esecuzione di test d'unità, può verificare il funzionamento delle funzioni integrate.
	
	\item[Server Administrator:] configura e gestisce un'istanza server dell'\emph{environment}.
	
\end{description}

\subsection{Organizzazione dei Casi d'uso}

\begin{figure}
\centering
\includegraphics[width=1.2\textwidth, angle=270]{Immagini/Capitolo2/UseCases/Vista-generale.png}
\caption{Vista generale dei casi d'uso principali}\label{fig:uc-vista-generale}
\end{figure}


I casi d'uso vengono organizzati in quattro gruppi differenti (\figurename~\ref{fig:uc-vista-generale}). Il gruppo di appartenenza di ognuno, indicato dal prefisso inserito nel nome, viene associato tenendo conto dell'attore principale al quale \emph{use case} fa riferimento. Il formato del nome di ogni caso d'uso seguirà quindi questa convenzione:
\begin{center}
UC[\emph{Attore}]-[\emph{Sezione}].[\emph{Gruppo}].[\emph{SottoGruppo}]: [\emph{Descrizione breve}]
\end{center}
dove:


\begin{description}
	\item[Attore] indica l'attore principale relazionato con il caso d'uso: \textbf{E} per \emph{Environment Developer}, \textbf{A} per \emph{Server Administrator}, \textbf{S} per \emph{Standalone Application}, \textbf{R} per \emph{Remote Application}, \textbf{D} per \emph{ES Developer} e \textbf{C} per \emph{ES Consumer}. Non è presente nessuno codice per l'attore \emph{Utente umano} in quanto non esiste alcun caso d'uso relativo esclusivamente ad esso.

	\item[Sezione] indica una delle sezioni principali presenti nella vista generale fornita in \figurename~\ref{fig:uc-vista-generale}.
	
	\item[Gruppo] e \textbf{SottoGruppo} esplicitano un'organizzazione gerarchica dei casi d'uso.

\end{description}

\pagebreak

\subsubsection{UCC-1: Utilizzo artefatto}

%UCC-1: Utilizzo artefatto

\begin{figure}
\centering
\includegraphics[width=1\textwidth]{Immagini/Capitolo2/UseCases/UCC-1.png}
\caption{Diagramma dei casi d'uso UCC-1}\label{fig:uc-ucc-1}
\end{figure}


\begin{itemize}
	\item \textbf{Attori:} ES Consumer
	\item \textbf{Scopo e descrizione:} un ES Consumer deve poter utilizzare un artefatto precedentemente creato, caricandolo e successivamente eseguendolo.
	\item \textbf{Pre-condizioni:} il software è correttamente configurato nel sistema
	\item \textbf{Post-condizioni:} il software ha eseguito tutte le operazioni desiderate
	\item \textbf{Flusso principale degli eventi:}
		\begin{enumerate}
			\item l'ES Consumer richiede al software di effettuare il caricamento di un artefatto (si veda caso d'uso \emph{UCC-1.1}).
			\item l'ES Consumer richiede al software di eseguire un artefatto (si veda caso d'uso \emph{UCC-1.2}).
		\end{enumerate}
\end{itemize}


\paragraph{UCC-1.1: Caricamento artefatto}

La descrizione comprende quelle dei casi d'uso \emph{UCC-1.1.1} e \emph{UCC-1.1.2} in quanto identici, a meno del formato con il quale l'artefatto viene fornito.

\begin{itemize}
	\item \textbf{Attori:} ES Consumer
	\item \textbf{Scopo e descrizione:} un ES Consumer deve avere la possibilità di caricare nel sistema un artefatto formalizzato in codice sorgente nativo o codice sorgente CLIPS
	\item \textbf{Pre-condizioni:} il software è correttamente configurato nel sistema
	\item \textbf{Post-condizioni:} il software ha eseguito le operazioni di caricamento ed il suo stato rispecchia quello descritto nell'artefatto
	\item \textbf{Flusso principale degli eventi:}
		\begin{enumerate}
			\item l'ES Consumer richiede al software di caricare un artefatto
			\begin{itemize}
				\item l'ES Consumer può fornire un artefatto in formato nativo
				\item l'ES Consumer può fornire un artefatto in formatto CLIPS
			\end{itemize}
			\item Il software analizza l'artefatto
			\item Il software esegue il caricamento dei costrutti contenuti nell'artefatto
		\end{enumerate}
	\item \textbf{Flusso alternativo:}
		\begin{enumerate}
			\setcounter{enumi}{2}
			\item se l'ES Consumer ha fornito un artefatto in formato non valido, il software notifica l'errore fornendo la motivazione e la posizione del problema
		\end{enumerate}
\end{itemize}

\paragraph{UCC-1.2: Esecuzione artefatto}

\begin{itemize}
	\item \textbf{Attori:} ES Consumer
	\item \textbf{Scopo e descrizione:} l'ES Consumer deve poter eseguire un artefatto caricato nel sistema. Se l'interazione lo richiede, l'ES Consumer deve essere in grado di fornire informazioni aggiuntive, osservare l'accadimento di un evento e osservare l'output fornito dal sistema
	\item \textbf{Pre-condizioni:} il software ha correttamente caricato un artefatto ed attende istruzioni.
	\item \textbf{Post-condizioni:} il software ha correttamente eseguito le operazioni previste dall'artefatto.
	\item \textbf{Flusso principale degli eventi:}
		\begin{enumerate}
			\item l'ES Consumer richiede l'esecuzione di un artefatto
			\item il sistema esegue le operazioni previste dall'artefatto (si vedano i casi d'uso \emph{UCC-1.2.1}, \emph{UCC-1.2.3})
			\item il sistema fornisce all'ES Consumer i risultati dell'esecuzione		
		\end{enumerate}
	\item \textbf{Flusso alternativo:} 
		\begin{enumerate}
			\setcounter{enumi}{1}
			\item se un'operazione specificata nell'artefatto genera errori, il sistema notifica l'errore all'ES Consumer fornendo informazioni sulla natura del problema.
			\item il sistema ritorna nello stato iniziale
		\end{enumerate}
\end{itemize}


\subparagraph{UCC-1.2.1: Fornisce input}

\begin{itemize}
	\item \textbf{Attori:} ES Consumer
	\item \textbf{Scopo e descrizione:} l'ES Consumer fornisce informazioni aggiuntive 
	\item \textbf{Pre-condizioni:} il sistema richiede informazioni aggiuntive per completare l'esecuzione di un artefatto
	\item \textbf{Post-condizioni:} il sistema prosegue l'esecuzione
	\item \textbf{Flusso principale degli eventi:}
		\begin{enumerate}
			\item il sistema richiede informazioni aggiuntive
			\item l'ES Consumer fornisce le informazioni nelle forme previste dall'artefatto
			\item il sistema valuta le informazioni fornite
			\item il sistema elabora le informazioni fornite
		\end{enumerate}
	\item \textbf{Flusso alternativo:} 
		\begin{enumerate}
			\setcounter{enumi}{2}
			\item il sistema notifica l'invalidità delle informazioni fornite
			\item lo scenario riprende dal punto 1 del flusso principale
		\end{enumerate}
\end{itemize}


\subparagraph{UCC-1.2.2: Visualizza output}

\begin{itemize}
	\item \textbf{Attori:} ES Consumer
	\item \textbf{Scopo e descrizione:} l'ES Consumer deve essere in grado di valutare l'output fornito dal sistema al termine di un'operazione
	\item \textbf{Pre-condizioni:} il sistema è pronto ad esegue un'operazione
	\item \textbf{Post-condizioni:} il sistema è pronto ad eseguire una nuova operazione
	\item \textbf{Flusso principale degli eventi:}
		\begin{enumerate}
			\item il sistema esegue un'operazione
			\item il sistema fornisce una visualizzazione del risultato
			\item l'ES Consumer valuta il risultato
		\end{enumerate}
\end{itemize}


\subparagraph{UCC-1.2.3: Osserva evento}

\begin{itemize}
	\item \textbf{Attori:} ES Consumer
	\item \textbf{Scopo e descrizione:} l'ES Consumer deve essere in grado di ricevere notifica del verificarsi di un evento
	\item \textbf{Pre-condizioni:} il sistema è pronto ad eseguire un'operazione
	\item \textbf{Post-condizioni:} il sistema è pronto ad eseguire un'operazione
	\item \textbf{Flusso principale degli eventi:}
		\begin{enumerate}
			\item l'ES Consumer richiede al sistema di eseguire un'operazione
			\item il sistema esegue un'operazione
			\item il sistema notifica all'ES Consumer il verificarsi di un evento
			\item l'ES Consumer valuta l'evento
		\end{enumerate}
\end{itemize}\pagebreak

\subsubsection{UCD-1: Creazione artefatto}

%UCD-1: Creazione artefatto

\begin{figure}
\centering
\includegraphics[width=1.1\textwidth]{Immagini/Capitolo2/UseCases/UCD-1.png}
\caption{Diagramma dei casi d'uso UCD-1}\label{fig:uc-ucd-1}
\end{figure}


\begin{itemize}
	\item \textbf{Attori:} ES Developer
	\item \textbf{Scopo e descrizione:} un ES Developer deve essere in grado di definire i costrutti necessari alla realizzazione di un artefatto.
	\item \textbf{Pre-condizioni:} il software fornisce almeno un linguaggio di specifica
	\item \textbf{Post-condizioni:} un artefatto è stato serializzato tramite il linguaggio di specifica
	\item \textbf{Flusso principale degli eventi:}
		\begin{enumerate}
			\item l'ES Developer definisce una serie di costrutti (si vedano i casi d'uso \emph{UCD-1.1}, \emph{UCD-1.2}, \emph{UCD-1.3}, \emph{UCD-1.4}, \emph{UCD-1.5}, \emph{UCD-1.6}).
			\begin{itemize}
				\item l'ES Developer può formalizzarlo in formato nativo
				\item l'ES Developer può formalizzarlo in formatto CLIPS
			\end{itemize}
			\item l'ES Consumer verifica la correttezza dell'artefatto (si veda il caso d'uso \emph{UCD-1.2}).
		\end{enumerate}
\end{itemize}


\paragraph{UCD-1.1: Definizione regola}

\begin{itemize}
	\item \textbf{Attori:} ES Developer
	\item \textbf{Scopo e descrizione:} un ES Developer deve essere in grado di definire una regola tramite un linguaggio di specifica fornito dal sistema.
	\item \textbf{Pre-condizioni:} il software fornisce almeno un linguaggio di specifica
	\item \textbf{Post-condizioni:} la definizione di una regola è stata aggiunta ad un artefatto
	\item \textbf{Flusso principale degli eventi:}
		\begin{enumerate}
			\item l'ES Developer definisce una regola
			\begin{itemize}
				\item l'ES Developer può formalizzarli in formato nativo
				\item l'ES Developer può formalizzarli in formatto CLIPS
			\end{itemize}
		\end{enumerate}
	\item \textbf{Flusso alternativo:}
		\begin{enumerate}
			\item se l'ES Developer ha definito precedentemente un modulo, l'ES Developer può definire una regola come componente del modulo.
		\end{enumerate}
\end{itemize}


\paragraph{UCD-1.2: Definizione template}

\begin{figure}[h]
\centering
\includegraphics[width=1\textwidth]{Immagini/Capitolo2/UseCases/UCD-1_2.png}
\caption{Diagramma dei casi d'uso UCD-1.2}\label{fig:uc-ucd-1.2}
\end{figure}

\begin{itemize}
	\item \textbf{Attori:} ES Developer
	\item \textbf{Scopo e descrizione:} l'ES Developer deve essere in grado di definire un template di fatti tramite un linguaggio di specifica fornito dal sistema.
	\item \textbf{Pre-condizioni:} il software fornisce almeno un linguaggio di specifica.
	\item \textbf{Post-condizioni:} la definizione di un template di fatti è stato aggiunto ad un artefatto
	\item \textbf{Flusso principale degli eventi:}
		\begin{enumerate}
			\item l'ES Developer definisce un template di fatti
			\item l'ES Developer definisce il formato del template di fatti (si vedano i casi d'uso \emph{UCD-1.2.1}, \emph{UCD-1.2.2})
		\end{enumerate}
	\item \textbf{Flusso alternativo:} 
		\begin{enumerate}
			\item se l'ES Developer ha definito precedentemente un modulo, l'ES Developer può definire un template di fatti come componente di un modulo
		\end{enumerate}
\end{itemize}

\subparagraph{UCD-1.2.1: Definizione slot}

\begin{itemize}
	\item \textbf{Attori:} ES Developer
	\item \textbf{Scopo e descrizione:} l'ES Developer deve essere in grado di definire il formato di uno slot appartenente ad un template di fatti
	\item \textbf{Pre-condizioni:} il software fornisce un linguaggio di specifica, l'ES developer ha definito un template di fatti
	\item \textbf{Post-condizioni:} il formato dello slot è stato aggiunto alla definizione di template di fatti
	\item \textbf{Flusso principale degli eventi:}
		\begin{enumerate}
			\item l'ES Developer definisce un template
			\item l'ES Developer definisce un nuovo slot fornendo il nome (stringa alfanumerica) da attribuire allo stesso
			\begin{enumerate}
				\item l'ES Developer può definire una restrizione di tipo al contenuto dello slot
				\item l'ES Developer può definire un valore di \emph{default} per lo slot
			\end{enumerate}
		\end{enumerate}
	\item \textbf{Flusso alternativo:} 
		\begin{enumerate}
			\setcounter{enumi}{1}
			\item se l'ES Developer ha fornito un nome già usato da un altro slot o multi-slot presente, il sistema rifiuta la definizione notificando un errore.
		\end{enumerate}
\end{itemize}

\subparagraph{UCD-1.2.2: Definizione multi-slot}

\begin{itemize}
	\item \textbf{Attori:} ES Developer
	\item \textbf{Scopo e descrizione:} l'ES Developer deve essere in grado di definire il formato di un multi-slot appartenente ad un template di fatti
	\item \textbf{Pre-condizioni:} il software fornisce un linguaggio di specifica, l'ES Developer ha definito un template di fatti
	\item \textbf{Post-condizioni:} il formato del multi-slot è stato aggiunto alla definizione di template di fatti
	\item \textbf{Flusso principale degli eventi:}
		\begin{enumerate}
			\item l'ES Developer definisce un template
			\item l'ES Developer definisce un nuovo multi-slot fornendo il nome (stringa alfanumerica) da attribuire allo stesso
			\begin{enumerate}
				\item l'ES Developer può definire delle restrizioni di tipo al contenuto dello slot
				\item l'ES Developer può definire dei valori di \emph{default} per il multi-slot
			\end{enumerate}
		\end{enumerate}
	\item \textbf{Flusso alternativo:} 
		\begin{enumerate}
			\setcounter{enumi}{1}
			\item se l'ES Developer ha fornito un nome già usato da un altro slot o multi-slot presente, il sistema rifiuta la definizione notificando un errore.
		\end{enumerate}
\end{itemize}


\paragraph{UCD-1.3: Definizione modulo}

\begin{figure}[h]
\centering
\includegraphics[width=1\textwidth]{Immagini/Capitolo2/UseCases/UCD-1_3.png}
\caption{Diagramma dei casi d'uso UCD-1.3}\label{fig:uc-ucd-1.3}
\end{figure}

\begin{itemize}
	\item \textbf{Attori:} ES Developer
	\item \textbf{Scopo e descrizione:} l'ES Developer deve essere in grado di definire un modulo tramite un linguaggio di specifica fornito dal sistema.
	\item \textbf{Pre-condizioni:} il software fornisce almeno un linguaggio di specifica.
	\item \textbf{Post-condizioni:} la definizione di un modulo è stata aggiunta ad un artefatto.
	\item \textbf{Flusso principale degli eventi:}
		\begin{enumerate}
			\item l'ES Developer definisce un modulo specificandone il nome
			\item l'ES Developer definisce una serie di costrutti da esportare (si veda il caso d'uso \emph{UCD-1.3.1})
			\item l'ES Developer definisce una serie di costrutti da importare da un altro modulo (si veda il caso d'uso \emph{UCD-1.3.2})
		\end{enumerate}
		
	\item \textbf{Flusso alternativo:} 
		\begin{enumerate}
			\setcounter{enumi}{0}
			\item se l'ES Developer fornisce un nome già utilizzato da un altro modulo, il sistema rifiuta la definizione e notifica l'errore.
			\item il sistema ritorna nello stato iniziale.
		\end{enumerate}
		
\end{itemize}

\subparagraph{UCD-1.3.1: Esporta costrutti}

\begin{itemize}
	\item \textbf{Attori:} ES Developer
	\item \textbf{Scopo e descrizione:} l'ES Developer deve essere in grado di specificare l'esportazione di un costrutto relativo al modulo tramite un linguaggio di specifica fornito dal sistema
	\item \textbf{Pre-condizioni:} il sistema fornisce un linguaggio di specifica, l'ES Developer ha definito un modulo
	\item \textbf{Post-condizioni:} il sistema aggiunge il nome del costrutto esportato in un insieme di elementi importabili da altri moduli
	\item \textbf{Flusso principale degli eventi:}
		\begin{enumerate}
			\item l'ES Developer definisce un modulo
			\item l'ES Developer definisce un costrutto appartenente al modulo come esportabile
				\begin{itemize}
					\item l'ES Developer può specificare il nome ed il tipo del costrutto
					\item l'ES Developer può specificare il tipo dei costrutti e un valore speciale per definire l'esportazione di tutti i costrutti del tipo indicato
					\item l'ES Developer può specificare un valore speciale per definire l'esportazione di tutti i costrutti di ogni tipo
					\item l'ES Developer può specificare il tipo ed un valore speciale per definire l'esclusione dei costrutti del tipo specificato dalla lista degli elementi esportati
				\end{itemize}
		\end{enumerate}
\end{itemize}


\subparagraph{UCD-1.3.2: Importa costrutti}

\begin{itemize}
	\item \textbf{Attori:} ES Developer
	\item \textbf{Scopo e descrizione:} l'ES Developer deve essere in grado di specificare l'importazione di un costrutto esportato da un modulo tramite un linguaggio di specifica fornito dal sistema
	\item \textbf{Pre-condizioni:} il sistema fornisce un linguaggio di specifica, l'ES Developer ha definito un modulo che esporta un costrutto, l'ES Developer ha definito un modulo
	\item \textbf{Post-condizioni:} il sistema aggiunge la definizione importata all'insieme di definizione del modulo.
	\item \textbf{Flusso principale degli eventi:}
		\begin{enumerate}
			\item l'ES Developer definisce un modulo
			\item l'ES Developer definisce un costrutto appartenente ad un secondo modulo come importabile.
				\begin{itemize}
					\item l'ES Developer può specificare il modulo, il nome ed il tipo del costrutto da importare
					\item l'ES Developer può specificare il modulo, il tipo ed un valore speciale per definite l'importazione di tutti i costrutti del tipo indicato dal modulo indicato
					\item l'ES Developer può specificare il modulo ed un valore speciale per definire l'importazione di tutti i costrutti dal modulo indicato
					
					\item l'ES Developer può specificare il modulo, il tipo ed un valore speciale per definire l'esclusione di tutti i costrutti del tipo indicato del modulo indicato dall'insieme di costrutti importati
				\end{itemize}
		\end{enumerate}
	\item \textbf{Flusso alternativo \#1:} 
		\begin{enumerate}
			\setcounter{enumi}{1}
			\item se l'ES Developer esegue un'importazione da un modulo non esistente, il sistema rifiuta la definizione e notifica l'errore.
			\item il sistema ritorna nello stato iniziale.
		\end{enumerate}
		
	\item \textbf{Flusso alternativo \#2:} 
		\begin{enumerate}
			\setcounter{enumi}{1}
			\item se l'ES Developer esegue un'importazione di un costrutto non ancora definito da un modulo che lo esporta preventivamente, il sistema rifiuta la definizione e notifica l'errore.
			\item il sistema ritorna nello stato iniziale.
		\end{enumerate}
		
\end{itemize}

\paragraph{UCD-1.4: Definizione fatti} % nome del caso d'uso Gruppo

\begin{figure}
\centering
\includegraphics[width=1.1\textwidth]{Immagini/Capitolo2/UseCases/UCD-1_4.png}
\caption{Diagramma dei casi d'uso UCD-1.4}\label{fig:uc-ucd-1.4}
\end{figure}

\begin{itemize}
	\item \textbf{Attori:} ES Developer
	\item \textbf{Scopo e descrizione:} l'ES Developer deve essere in grado di definire un gruppo di fatti iniziali tramite un linguaggio di specifica fornito dal sistema
	\item \textbf{Pre-condizioni:} il software fornisce almeno un linguaggio di specifica
	\item \textbf{Post-condizioni:} la definizione del gruppo di fatti è stata aggiunta ad un artefatto
	\item \textbf{Flusso principale degli eventi:}
		\begin{enumerate}
			\item l'ES Developer definisce un gruppo di fatti specificando il nome
			\item l'ES Developer definisce una serie di fatti da associare al gruppo (si vedano i casi d'uso \emph{UCD-1.4.1} e \emph{UCD-1.4.2})
		\end{enumerate}
	\item \textbf{Flusso alternativo:}
		\begin{enumerate}
			\item se l'ES Developer ha definito precedentemente un modulo, l'ES Developer può definire il gruppo di fatti come componente del modulo.
			\item lo scenario prosegue dal punto 2 del flusso principale
		\end{enumerate}
\end{itemize}

\subparagraph{UCD-1.4.1: Definizione ordered-fact} % nome del caso d'uso SottoGruppo

\begin{itemize}
	\item \textbf{Attori:} ES Developer
	\item \textbf{Scopo e descrizione:}  l'ES Developer deve essere in grado di definire un fatto in notazione \emph{ordered-fact} tramite un linguaggio di specifica fornito dal sistema
	\item \textbf{Pre-condizioni:} il software fornisce almeno un linguaggio di specifica, l'ES Developer ha definito un gruppo di fatti iniziali
	\item \textbf{Post-condizioni:} il fatto in notazione \emph{ordered-fact} è aggiunto al gruppo di fatti iniziali
	\item \textbf{Flusso principale degli eventi:}
		\begin{enumerate}
			\item l'ES Developer definisce un gruppo di fatti iniziali
			\item l'ES Developer definisce un fatto in notazione \emph{ordered-fact} specificando l'elenco di componenti in sequenza
				\begin{itemize}
					\item l'ES Developer può specificare una sequenza di componenti di lunghezza arbitraria
					\item l'ES Developer può specificare una sequenza di componenti di tipo non omogeneo
				\end{itemize}
		\end{enumerate}
	\item \textbf{Flusso alternativo:}
		\begin{enumerate}
			\setcounter{enumi}{1}
			\item l'ES Developer definisce un fatto in notazione \emph{ordered-fact} con componenti identiche a quelle di un fatto specificato in precedenza
			\item il sistema ignora la definizione
		\end{enumerate}
\end{itemize}


\subparagraph{UCD-1.4.2: Definizione template-fact} % nome del caso d'uso SottoGruppo

\begin{itemize}
	\item \textbf{Attori:} ES Developer
	\item \textbf{Scopo e descrizione:}  l'ES Developer deve essere in grado di definire un fatto in notazione \emph{template-fact} tramite un linguaggio di specifica fornito dal sistema
	\item \textbf{Pre-condizioni:} il software fornisce almeno un linguaggio di specifica, l'ES Developer ha definito un gruppo di fatti iniziali
	\item \textbf{Post-condizioni:} il fatto in notazione \emph{template-fact} è aggiunto al gruppo di fatti iniziali
	\item \textbf{Flusso principale degli eventi:}
		\begin{enumerate}
			\item l'ES Developer definisce un gruppo di fatti iniziali
			\item l'ES Developer definisce un fatto in notazione \emph{template-fact} specificando il nome del \emph{template} di riferimento
			\item l'ES Developer definisce i valori degli slot (risp. multi-slot)
				\begin{itemize}
					\item l'ES Developer può specificare una sequenza di componenti di lunghezza arbitraria per i \emph{multi-slot}
					\item l'ES Developer può specificare un elemento per gli \emph{slot}
				\end{itemize}
			\item il sistema completa la definizione del fatto associando i valori di default agli \emph{slot} (risp. \emph{multi-slot}) non definiti.
		\end{enumerate}
	\item \textbf{Flusso alternativo \#1:}
		\begin{enumerate}
			\setcounter{enumi}{1}
			\item l'ES Developer fornisce un nome di template non valido
			\item il sistema rifiuta la definizione e notifica l'errore
			\item il sistema torna allo stato iniziale
		\end{enumerate}
	\item \textbf{Flusso alternativo \#2:}
		\begin{enumerate}
			\setcounter{enumi}{2}
			\item l'ES Developer assegna un valore ad uno \emph{slot} (risp. \emph{multi-slot}) non definito nella definizione del \emph{template}.
			\item il sistema rifiuta la definizione e notifica l'errore
			\item il sistema torna allo stato iniziale
		\end{enumerate}
\end{itemize}


\paragraph{UCD-1.5: Definizione variabile globale} % nome del caso d'uso Gruppo

\begin{itemize}
	\item \textbf{Attori:} ES Developer
	\item \textbf{Scopo e descrizione:}  l'ES Developer deve essere in grado di definire una variabile globale tramite un linguaggio di specifica.
	\item \textbf{Pre-condizioni:} il software fornisce almeno un linguaggio di specifica
	\item \textbf{Post-condizioni:} la definizione della variabile globale è stata aggiunta ad un artefatto
	\item \textbf{Flusso principale degli eventi:}
		\begin{enumerate}
			\item l'ES Developer definisce una variabile globale, indicando nome e valore iniziale
			\item il sistema valuta la definizione
		\end{enumerate}
	\item \textbf{Flusso alternativo \#1:} 
		\begin{enumerate}
			\setcounter{enumi}{1}
			\item se l'ES Developer ha indicato come nome per la variabile globale uno già utilizzato, il valore iniziale fornito viene sostituito a quello della definizione precedente
			\item lo scenario prosegue dal punto 2 del flusso principale
		\end{enumerate}
	\item \textbf{Flusso alternativo \#2:} 
		\begin{enumerate}
			\item se l'ES Developer ha definito precedentemente un modulo, l'ES Developer può definire la variabile globale come componente del modulo.
			\item lo scenario prosegue dal punto 2 del flusso principale
		\end{enumerate}
\end{itemize}

\paragraph{UCD-1.6: Definizione funzione} % nome del caso d'uso Gruppo

\begin{itemize}
	\item \textbf{Attori:} ES Developer
	\item \textbf{Scopo e descrizione:} l'ES Developer deve essere in grado di definire una funzione tramite un linguaggio di specifica
	\item \textbf{Pre-condizioni:} il software fornisce almeno un linguaggio di specifica
	\item \textbf{Post-condizioni:} la definizione della funzione è stata aggiunta ad un artefatto
	\item \textbf{Flusso principale degli eventi:}
		\begin{enumerate}
			\item l'ES Developer definisce una funzione utente, indicando nome, parametri e corpo della funzione
			\item il sistema valuta il nome della funzione
		\end{enumerate}
	\item \textbf{Flusso alternativo \#1:}
		\begin{enumerate}
			\setcounter{enumi}{1}
			\item se il nome fornito dall'ES Developer è già utilizzato da una definizione di funzione (di sistema o utente), il sistema rifiuta la definizione e notifica l'errore
			\item il sistema ritorna allo stato iniziale
		\end{enumerate}
	\item \textbf{Flusso alternativo \#2:}
		\begin{enumerate}
			\item se l'ES Developer ha definito precedentemente un modulo, l'ES Developer può definire la funzione utente come componente del modulo
			\item lo scenario procede dal punto 2 del flusso principale
		\end{enumerate}
\end{itemize}

\pagebreak

\subsubsection{UCD-2: Testa artefatto}

%UCD-2: Testa artefatto

\begin{figure}
\centering
\includegraphics[width=1.1\textwidth]{Immagini/Capitolo2/UseCases/UCD-2.png}
\caption{Diagramma dei casi d'uso UCD-2}\label{fig:uc-ucd-2}
\end{figure}


\begin{itemize}
	\item \textbf{Attori:} ES Developer
	\item \textbf{Scopo e descrizione:} un ES Developer deve essere in grado di testare l'artefatto prodotto valutando un registro del funzionamento dello stesso e le modalità con le quali le definizioni appartenenti all'artefatto vengono interpretate.
	\item \textbf{Pre-condizioni:} l'ES Developer ha realizzato un artefatto, l'artefatto è stato caricato nel sistema. Il sistema attende istruzioni
	\item \textbf{Post-condizioni:} il funzionamento dell'artefatto è stato valutato
	\item \textbf{Flusso principale degli eventi:}
		\begin{enumerate}
			\item l'ES Developer richiede al sistema informazioni per valutare l'artefatto.
			\begin{itemize}
				\item l'ES Developer può richiedere informazioni sull'attività del sistema (si veda caso d'uso \emph{UCD-2.1})
				\item l'ES Developer può richiedere informazioni su come l'artefatto viene interpretato (si veda caso d'uso \emph{UCD-2.2}).
			\end{itemize}
			\item il sistema fornisce le informazioni richieste.
		\end{enumerate}
\end{itemize}


\paragraph{UCD-2.1: Attiva/Disattiva tracciamento}

\begin{figure}
\centering
\includegraphics[width=1.1\textwidth]{Immagini/Capitolo2/UseCases/UCD-2_1.png}
\caption{Diagramma dei casi d'uso UCD-2.1}\label{fig:uc-ucd-2.1}
\end{figure}

La descrizione comprende quella dei casi d'uso \emph{UCD-2.1.1}, \emph{UCD-2.1.2}, \emph{UCD-2.1.3} in quanto identici, a meno del tipo di informazioni che vengono tracciate.

\begin{itemize}
	\item \textbf{Attori:} ES Developer
	\item \textbf{Scopo e descrizione:} un ES Developer deve essere in grado di modificare le impostazioni riguardo al tracciamento di regole, attivazioni e modifiche alla \emph{working-memory}.
	\item \textbf{Pre-condizioni:} ES Developer ha caricato un artefatto nel sistema. Il sistema attende istruzioni
	\item \textbf{Post-condizioni:} le impostazioni di tracciamento risultano modificate
	\item \textbf{Flusso principale degli eventi:}
		\begin{enumerate}
			\item l'ES Developer modifica le impostazioni riguardanti il tracciamento
			\begin{itemize}
				\item l'ES Developer può attivare (risp. disattivare) il tracciamento delle regole eseguite
				\item l'ES Developer può attivare (risp. disattivare) il tracciamento delle attivazioni disponibili
				\item l'ES Developer può attivare (risp. disattivare) il tracciamento delle modifiche alla \emph{working-memory}				
			\end{itemize}
			\item il sistema modifica le impostazioni di tracciamento dell'elemento indicato
		\end{enumerate}
	\item \textbf{Flusso alternativo:}
		\begin{enumerate}
			\item l'ES Developer richiede l'attivazione (risp. disattivazione) di un tracciamento già attivo (risp. non attivo).
			\item il sistema ignora la richiesta
		\end{enumerate}
\end{itemize}


\paragraph{UCD-2.2: Visualizzazione grafico regole}

\begin{itemize}
	\item \textbf{Attori:} ES Developer
	\item \textbf{Scopo e descrizione:} un ES Developer deve essere in grado di richiedere informazioni riguardanti le modalità con le quali il sistema interpreta una regola.
	\item \textbf{Pre-condizioni:} ES Developer ha caricato un artefatto nel sistema. Il sistema attende istruzioni
	\item \textbf{Post-condizioni:} Il sistema attende istruzioni
	\item \textbf{Flusso principale degli eventi:}
		\begin{enumerate}
			\item l'ES Developer richiede informazioni riguardanti l'interpretazione di una regola
			\begin{itemize}
				\item l'ES Developer può indicare una regola specificandone il nome
				\item l'ES Developer può indicare un insieme di regole specificandone i nomi
			\end{itemize}
			\item il sistema fornisce le informazioni richieste
		\end{enumerate}
	\item \textbf{Flusso alternativo \#1:}
		\begin{enumerate}
			\item l'ES Developer specifica un nome di regola non valido
			\item il sistema ignora la richiesta, notificando l'errore
		\end{enumerate}
	\item \textbf{Flusso alternativo \#2:}
		\begin{enumerate}
			\item l'ES Developer inserisce un nome non valido specificando un insieme di regole
			\item il sistema ignora la richiesta, notificando l'errore
		\end{enumerate}
\end{itemize}

\pagebreak

\subsubsection{UCE-1: Modifica funzioni di sistema}

%UCE-1: Modifica funzioni di sistema

Nel proseguo dei paragrafi relativi ai casi d'uso \emph{UCE}, il nome dell'attore \emph{Environment Developer} verrà abbreviato con la sigla \emph{ED}.

\begin{figure}
\centering
\includegraphics[width=1.1\textwidth]{Immagini/Capitolo2/UseCases/UCE-1.png}
\caption{Diagramma dei casi d'uso UCE-1}\label{fig:uc-uce-1}
\end{figure}


\begin{itemize}
	\item \textbf{Attori:} Environment Developer (\emph{ED})
	\item \textbf{Scopo e descrizione:} l'ED deve essere in grado di modificare il set di funzioni di sistema incluse nel software
	\item \textbf{Pre-condizioni:} il software fornisce meccanismi per estendere il set di funzioni di sistema
	\item \textbf{Post-condizioni:} il set di funzioni di sistema risulta modificato
	\item \textbf{Flusso principale degli eventi:}
		\begin{enumerate}
			\item l'ED esegue una modifica al set di funzioni di sistema.
			\begin{itemize}
				\item l'ED può rimuovere la definizione di una funzione esistente (si veda caso d'uso \emph{UCE-1.2})
				\item l'ED può aggiungere una nuova definizione di funzione di sistema (si veda caso d'uso \emph{UCE-1.1}).
			\end{itemize}
			\item il sistema carica il set modificato di funzioni sistema.
		\end{enumerate}
\end{itemize}


\paragraph{UCE-1.1: Aggiunge funzione}

La descrizione comprende quella del caso d'uso \emph{UCE-1.1.1}

\begin{itemize}
	\item \textbf{Attori:} ED
	\item \textbf{Scopo e descrizione:} un ED deve essere in grado di aggiungere una nuova funzione al set di funzioni di sistema.
	\item \textbf{Pre-condizioni:} il sistema è stato configurato correttamente. Il sistema non è attivo
	\item \textbf{Post-condizioni:} il sistema è attivo e una nuova funzione è disponibile nel set di funzioni di sistema
	\item \textbf{Flusso principale degli eventi:}
		\begin{enumerate}
			\item l'ED definisce il corpo di una nuova funzione di sistema (\emph{UCE-1.1.1})
			\item l'ED definisce la firma di una nuova funzione di sistema
				\begin{itemize}
					\item l'ED specifica un nome di funzione
					\item l'ED specifica un numero di parametri accettati (minimi, massimi, esatti)
					\item l'ED specifica il tipo dei parametri accettati
					\item l'ED specifica il tipo dei risultati
				\end{itemize}
			\item l'ED avvia il software
			\item l'ED richiede al software il caricamento della nuova funzione
				\begin{itemize}
					\item l'ED indica il file contenente il corpo della funzione
					\item l'ED indica la posizione della firma della funzione
				\end{itemize}
			\item la richiesta è valutata dal sistema
			\item la nuova definizione viene resa disponibile
		\end{enumerate}
	\item \textbf{Flusso alternativo \#1:}
		\begin{enumerate}
			\setcounter{enumi}{4}
			\item se l'ED ha specificato un nome di funzione già utilizzato, il sistema ignora la nuova definizione, notificando l'errore
		\end{enumerate}

	\item \textbf{Flusso alternativo \#2:}
		\begin{enumerate}
			\setcounter{enumi}{1}
			\item l'ED definisce la firma della nuova funzione indicando solo il nome e il tipo dei risultati della funzione
			\item \emph{analogo al punto 3 del flusso principale}
			\item \emph{analogo al punto 4 del flusso principale}
			\item \emph{analogo al punto 5 del flusso principale}						
			\item la firma viene integrata automaticamente dal sistema in modo che la funzione accetti un numero ed un tipo arbitrario di valori
			\item lo scenario prosegue con il punto 6 del flusso principale
		\end{enumerate}

	
\end{itemize}


\subparagraph{UCE-1.1.2: Modifica manifesto funzioni} % nome del caso d'uso SottoGruppo

\begin{itemize}
	\item \textbf{Attori:} ED
	\item \textbf{Scopo e descrizione:} ED deve essere in grado di apportare una modifica permanente al set di funzioni di sistema modificando un registro contenente l'elenco di funzioni di sistema
	\item \textbf{Pre-condizioni:} l'ED vuole rendere permanente l'aggiunta o la rimozione di una funzione di sistema. Il software non è attivo.
	\item \textbf{Post-condizioni:} il software è attivo, la modifica apportata al registro viene evidenziata nel set di funzioni di sistema disponibili
	\item \textbf{Flusso principale degli eventi:}
		\begin{enumerate}
			\item l'ED modifica il registro contenente l'elenco delle funzioni di sistema
				\begin{itemize}
					\item l'ED può aggiungere una nuova funzione all'elenco specificando il file contenente il corpo e la posizione della firma della funzione
					\item l'ED può rimuovere una funzione eliminandone le informazioni dall'elenco
				\end{itemize}
			\item l'ED salva le modifiche al registro
			\item l'ED avvia il sistema
			\item le modifiche vengono valutate dal sistema
			\item il set di funzioni disponibili viene modificato
			\item l'inizializzazione del software prosegue
		\end{enumerate}
	\item \textbf{Flusso alternativo:} 
		\begin{enumerate}
			\setcounter{enumi}{4}
			\item se l'ED ha apportato modifiche non valide al registro delle funzioni di sistema, il software notifica l'errore
			\item l'inizializzazione del software viene arrestata
		\end{enumerate}
\end{itemize}


\paragraph{UCE-1.2: Rimuove funzione}

\begin{itemize}
	\item \textbf{Attori:} ED
	\item \textbf{Scopo e descrizione:} ED deve essere in grado di rimuovere una funzione di sistema da quelle disponibili.
	\item \textbf{Pre-condizioni:} il software è configurato correttamente, il software non è attivo.
	\item \textbf{Post-condizioni:} Il software è attivo, la funzione di sistema non è presente fra quelle disponibili
	\item \textbf{Flusso principale degli eventi:}
		\begin{enumerate}
			\item l'ED modifica il manifesto delle funzioni di sistema rimuovendo le informazioni riguardanti la funzione (si guardi caso il d'uso \emph{UCE-1.1.2})
		\end{enumerate}
\end{itemize}

\pagebreak

\subsubsection{UCE-2: Monitorizza evento}

%UCE-2: Monitorizza evento

\begin{figure}
\centering
\includegraphics[width=1.1\textwidth]{Immagini/Capitolo2/UseCases/UCE-2.png}
\caption{Diagramma dei casi d'uso UCE-2}\label{fig:uc-uce-2}
\end{figure}


\begin{itemize}
	\item \textbf{Attori:} ED
	\item \textbf{Scopo e descrizione:} l'ED deve essere in grado di specificare un nuovo evento e definire un comportamento da associare al verificarsi dell'evento
	\item \textbf{Pre-condizioni:} il software fornisce meccanismi per la definizione e la notifica di un evento
	\item \textbf{Post-condizioni:} il comportamento specificato dall'ED viene eseguito al verificarsi dell'evento atteso
	\item \textbf{Flusso principale degli eventi:}
		\begin{enumerate}
			\item l'ED specifica un nuovo evento
			\item l'ED specifica un \emph{listener} che realizzi un comportamento al verificarsi di un evento (si guardi il caso d'uso \emph{UCE-2.1})
			\item l'ED avvia il sistema
			\item l'ED avvia un'elaborazione
			\item il verificarsi dell'evento provoca l'esecuzione del comportamento definito
		\end{enumerate}
\end{itemize}


\paragraph{UCE-2.1: Definisce listener}

\begin{itemize}
	\item \textbf{Attori:} ED
	\item \textbf{Scopo e descrizione:} un ED deve essere in grado di specificare un comportamento da eseguire al verificarsi di un evento
	\item \textbf{Pre-condizioni:} il sistema è stato configurato correttamente. Il sistema non è attivo
	\item \textbf{Post-condizioni:} il sistema è attivo e un comportamento è associato all'accadere di un evento
	\item \textbf{Flusso principale degli eventi:}
		\begin{enumerate}
			\item l'ED definisce un comportamento nella forma di un entità \emph{listener}
			\item l'ED registra il \emph{listener} nel sistema
		\end{enumerate}
	
\end{itemize}

\pagebreak

\subsubsection{UCE-3: Aggiunge CRS}

%UCE-3: Aggiunge CRS

\begin{figure}
\centering
\includegraphics[width=1.1\textwidth]{Immagini/Capitolo2/UseCases/UCE-3.png}
\caption{Diagramma dei casi d'uso UCE-3}\label{fig:uc-uce-3}
\end{figure}


\begin{itemize}
	\item \textbf{Attori:} ED
	\item \textbf{Scopo e descrizione:} l'ED deve essere in grado di definire ed aggiungere una nuova \emph{strategia di risoluzione dei conflitti} (CRS)
	\item \textbf{Pre-condizioni:} il software fornisce meccanismi per la definizione e l'aggiunta di CRS, il sistema non è attivo
	\item \textbf{Post-condizioni:} il sistema è attivo e la una nuova CRS è disponibile
	\item \textbf{Flusso principale degli eventi:}
		\begin{enumerate}
			\item l'ED definisce una nuova CRS attraverso i meccanismi previsti dal sistema (si guardi il caso d'uso \emph{UCE-3.1})
			\item l'ED aggiunge la nuova CRS al sistema
			\item l'ED avvia il sistema
		\end{enumerate}
\end{itemize}


\paragraph{UCE-3.1: Definisce CRS}

\begin{itemize}
	\item \textbf{Attori:} ED
	\item \textbf{Scopo e descrizione:} l'ED deve essere in grado di definire una nuova CRS
	\item \textbf{Pre-condizioni:} il sistema è stato configurato correttamente. Il sistema non è attivo. Il sistema offre meccanismi per la definizione di una nuova CRS.
	\item \textbf{Post-condizioni:} una nuova CRS è disponibile per l'integrazione al sistema
	\item \textbf{Flusso principale degli eventi:}
		\begin{enumerate}
			\item l'ED definisce una nuova CRS
			\item l'ED realizza la CRS in una unità di elaborazione separata e raggiungibile al software
		\end{enumerate}
	
\end{itemize}

\pagebreak

\subsubsection{UCE-4: Testa environment}

%UCE-4: Testa environment

\begin{figure}
\centering
\includegraphics[width=1.1\textwidth]{Immagini/Capitolo2/UseCases/UCE-4.png}
\caption{Diagramma dei casi d'uso UCE-4}\label{fig:uc-uce-4}
\end{figure}

La descrizione comprende quella del caso d'uso \emph{UCE-4.1}.

\begin{itemize}
	\item \textbf{Attori:} ED
	\item \textbf{Scopo e descrizione:} l'ED deve essere in grado di verificare il funzionamento dell'environment utilizzando dei test d'unità. L'ED deve avere la possibilità di realizzare dei test d'unità per le funzionalità aggiunte
	\item \textbf{Pre-condizioni:} il software fornisce meccanismi per l'esecuzione di test d'unità in grado di verificare il comportamento dei componenti di sistema. Il software fornisce un framework per la definizione di nuovi test d'unità per le funzionalità introdotte
	\item \textbf{Post-condizioni:} l'ED ottiene dal software un riepilogo sul funzionamento delle componenti
	\item \textbf{Flusso principale degli eventi:}
		\begin{enumerate}
			\item l'ED avvia l'esecuzione dei test d'unità (\emph{UCE-4.1})
			\item l'ED ottiene un riepilogo dei risultati dei test
				\begin{itemize}
					\item il riepilogo indica eventuali test falliti
					\item il riepilogo indica il motivo del fallimento dei test
				\end{itemize}
		\end{enumerate}
\end{itemize}


\paragraph{UCE-4.1: Creazione unittest}

\begin{itemize}
	\item \textbf{Attori:} ED
	\item \textbf{Scopo e descrizione:} l'ED deve avere la possibilità di definire ed eseguire test d'unità in grado di verificare il corretto funzionamento delle caratteristiche aggiunte
	\item \textbf{Pre-condizioni:} il sistema è stato configurato correttamente. Il software offre meccanismi per l'integrazione di nuovi test d'unità a quelli forniti con il sistema
	\item \textbf{Post-condizioni:} un nuovo test d'unità è integrato con quelli forniti dal sistema
	\item \textbf{Flusso principale degli eventi:}
		\begin{enumerate}
			\item l'ED realizza un nuovo test d'unità
			\item l'ED registra il nuovo test d'unità insieme a quelli forniti dal software
		\end{enumerate}
\end{itemize}

\pagebreak

\subsubsection{UCA-1: Avvia Server}

%UCA-1: Avvia server

\begin{itemize}
	\item \textbf{Attori:} Server Administrator
	\item \textbf{Scopo e descrizione:} il Server Administrator deve essere in inizializzare un'istanza server del software
	\item \textbf{Pre-condizioni:} il sistema è correttamente configurato. Il software è correttamente configurato
	\item \textbf{Post-condizioni:} un'istanza server del software è attiva e attende connessioni
	\item \textbf{Flusso principale degli eventi:}
		\begin{enumerate}
			\item il Server Administrator avvia un'istanza del software
			\item la configurazione del software viene valutata (si guardi caso d'uso \emph{UCA-3})
		\end{enumerate}
	\item \textbf{Flusso alternativo \#1:}
		\begin{enumerate}
			\setcounter{enumi}{1}
			\item se una specifica di configurazione non è disponibile, il software notifica l'errore
			\item vengono utilizzati dei parametri di default
		\end{enumerate}
	\item \textbf{Flusso alternativo \#2:}
		\begin{enumerate}
			\setcounter{enumi}{1}
			\item se la specifica di configurazione non è valida, il software notifica l'errore
			\item l'esecuzione viene terminata
		\end{enumerate}
\end{itemize}
\pagebreak

\subsubsection{UCA-2: Arresta Server}

%UCA-2: Arresta Server

\begin{itemize}
	\item \textbf{Attori:} Server Administrator
	\item \textbf{Scopo e descrizione:} il Server Administrator deve essere in terminare l'esecuzione di un'istanza del software
	\item \textbf{Pre-condizioni:} un'istanza del software è attiva
	\item \textbf{Post-condizioni:} l'istanza del software non è più attiva
	\item \textbf{Flusso principale degli eventi:}
		\begin{enumerate}
			\item il Server Administrator richiede l'arresto del software
			\item tutte le sessioni vengono chiuse
			\item il software viene arrestato
		\end{enumerate}
	\item \textbf{Flusso alternativo:}
		\begin{enumerate}
			\setcounter{enumi}{1}
			\item se si verifica un errore durante la chiusura di una sessione (o un tempo limite per l'operazione viene superato), la sessione viene distrutta forzatamente
			\item lo scenario prosegue dal punto 3 del flusso principale
		\end{enumerate}
\end{itemize}
\pagebreak

\subsubsection{UCA-3: Configura Server}

%UCA-3: Configura Server

\begin{figure}
\centering
\includegraphics[width=1.1\textwidth]{Immagini/Capitolo2/UseCases/UCA-3.png}
\caption{Diagramma dei casi d'uso UCA-3}\label{fig:uc-uca-3}
\end{figure}

\begin{itemize}
	\item \textbf{Attori:} Server Administrator
	\item \textbf{Scopo e descrizione:} il Server Administrator deve avere la possibilità di configurare le funzioni del software MyCLIPS utilizzabili dai client, i servizi offerti dal server e i parametri di base del modulo server attraverso un file di configurazione
	\item \textbf{Pre-condizioni:} il sistema offre un formato di specifica delle configurazioni e un meccanismo di caricamento delle stesse
	\item \textbf{Post-condizioni:} il funzionamento del software rispecchia i parametri definiti del file di configurazione
	\item \textbf{Flusso principale degli eventi:}
		\begin{enumerate}
			\item il Server Administrator apre il file di configurazione attraverso un programma di \emph{text-editing}
			\item il Server Administrator modifica i parametri di configurazione del server
				\begin{itemize}
					\item il Server Administrator modifica le funzioni di MyCLIPS utilizzabili dai client (si guardi il caso d'uso \emph{UCA-3.3})
					\item il Server Administrator modifica i servizi attivi offerti dal server (si guardi il caso d'uso \emph{UCA-3.2})
					\item il Server Administrator modifica i parametri di base del server (si guardi il caso d'uso \emph{UCA-3.1})
				\end{itemize}
			\item il Server Administrator salva le modifiche
		\end{enumerate}
\end{itemize}


\paragraph{UCA-3.1: Configurazione parametri server}

\begin{itemize}
	\item \textbf{Attori:} Server Administrator
	\item \textbf{Scopo e descrizione:} il Server Administrator deve avere la possibilità di modificare i parametri di configurazione di base del server
	\item \textbf{Pre-condizioni:} \emph{nessuna}
	\item \textbf{Post-condizioni:} i parametri di configurazione risultano modificati secondo le aspettative del Server Administrator
	\item \textbf{Flusso principale degli eventi:}
		\begin{enumerate}
			\item il Server Administrator modifica i parametri di base del server. Può configurare:
				\begin{itemize}
					\item l'indirizzo di ascolto (\emph{bind-address})
					\item la porta di ascolto (\emph{bind-port})
					\item la creazione di un registro delle richieste (\emph{log-requests})
					\item il livello di dettaglio del registro delle richieste (\emph{log-level})
				\end{itemize}
		\end{enumerate}
\end{itemize}


\paragraph{UCA-3.2: Definizione servizi}

La descrizione comprende anche quella del caso d'uso \emph{UCA-3.2.1}.

\begin{itemize}
	\item \textbf{Attori:} Server Administrator
	\item \textbf{Scopo e descrizione:} il Server Administrator deve avere la possibilità di definire i servizi attivi offerti dal server e configurare i parametri di funzionamento dei servizi
	\item \textbf{Pre-condizioni:} \emph{nessuna}
	\item \textbf{Post-condizioni:} i parametri di configurazione risultano modificati secondo le aspettative del Server Administrator
	\item \textbf{Flusso principale degli eventi:}
		\begin{enumerate}
			\item il Server Administrator modifica i servizi attivi offerti dal server specificando:
				\begin{itemize}
					\item l'interfaccia offerta dal servizio
					\item un'implementazione del servizio
				\end{itemize}
			\item il Server Administrator specifica i parametri relativi ai servizi attivi (\emph{UCA-3.2.1)}: il formato, il tipo e i valori delle configurazioni sono definite dalla specifica implementazione del servizio
		\end{enumerate}
\end{itemize}





\pagebreak

\subsubsection{UCS-1: Analisi dei risultati}

%UCS-1: Analisi dei risultati

\begin{figure}
\centering
\includegraphics[width=1.1\textwidth]{Immagini/Capitolo2/UseCases/UCS-1.png}
\caption{Diagramma dei casi d'uso UCS-1}\label{fig:uc-ucs-1}
\end{figure}

\begin{itemize}
	\item \textbf{Attori:} Standalone Application
	\item \textbf{Scopo e descrizione:} la Standalone Application deve avere la possibilità di accedere alle informazioni riguardanti lo stato finale di un'elaborazione eseguita dal software e ricevere notifica del verificarsi di eventi durante l'elaborazione
	\item \textbf{Pre-condizioni:} la Standalone Application ha eseguito con successo il caso d'uso \emph{UCC-1.2}
	\item \textbf{Post-condizioni:} la Standalone Application ha ottenuto le informazioni necessarie
	\item \textbf{Flusso principale degli eventi:}
		\begin{enumerate}
			\item la Standalone Application richiede informazioni sullo stato terminale di elaborazione del software
				\begin{itemize}
					\item la Standalone Application richiede informazioni sullo stato della \emph{working-memory} (si guardi il caso d'uso \emph{UCS-1.1})
					\item la Standalone Application richiede informazioni sullo stato delle attivazioni ancora presenti (si guardi il caso d'uso \emph{UCS-1.2})
					\item la Standalone Application valuta il verificarsi di un evento (si guardi il caso d'uso \emph{UCS-1.3})
				\end{itemize}
			\item la Standalone Application ottiene le informazioni richieste
		\end{enumerate}
\end{itemize}


\paragraph{UCS-1.1: Analisi della Working-Memory}

\begin{itemize}
	\item \textbf{Attori:} Standalone Application
	\item \textbf{Scopo e descrizione:} la Standalone Application deve avere accesso alle informazioni presenti nella \emph{Working Memory}
	\item \textbf{Pre-condizioni:} la Standalone Application ha eseguito con successo il caso d'uso \emph{UCC-1.2}
	\item \textbf{Post-condizioni:} la Standalone Application ottiene le informazioni sui fatti presenti nella \emph{Working Memory}
	\item \textbf{Flusso principale degli eventi:}
		\begin{enumerate}
			\item la Standalone Application richiede una vista dei dati presenti nella \emph{Working Memory}
				\begin{itemize}
					\item la Standalone Application può limitare la ricerca solo ai fatti relativi ad un sottogruppo dei moduli specificati
					\item la Standalone Application può richiedere tutti i fatti presenti nella \emph{Working Memory}
				\end{itemize}
			\item la Standalone Application ottiene l'elenco dei fatti richiesti
		\end{enumerate}
\end{itemize}


\paragraph{UCS-1.2: Analisi attivazioni}

\begin{itemize}
	\item \textbf{Attori:} Standalone Application
	\item \textbf{Scopo e descrizione:} la Standalone Application deve avere accesso alle informazioni presenti nell'agenda delle attivazioni
	\item \textbf{Pre-condizioni:} la Standalone Application ha eseguito con successo il caso d'uso \emph{UCC-1.1}
	\item \textbf{Post-condizioni:} la Standalone Application ottiene l'elenco di attivazioni disponibili nell'agenda delle attivazioni nelle modalità che ha indicato
	\item \textbf{Flusso principale degli eventi:}
		\begin{enumerate}
			\item la Standalone Application richiede una vista sulle attivazioni presenti nell'agenda delle attivazioni.
				\begin{itemize}
					\item la Standalone Application può limitare la ricerca solo alle attivazioni relative ad un sottogruppo dei moduli specificati
					\item la Standalone Application può richiedere l'intera lista delle attivazioni presenti in agenda
				\end{itemize}
			\item la Standalone Application ottiene l'elenco delle attivazioni richieste
		\end{enumerate}
\end{itemize}


\paragraph{UCS-1.3: Analisi eventi}

\begin{itemize}
	\item \textbf{Attori:} Standalone Application
	\item \textbf{Scopo e descrizione:} la Standalone Application deve ricevere notifica del verificarsi di un evento per il quale ha espresso interesse nelle modalità previste dal software
	\item \textbf{Pre-condizioni:} la Standalone Application ha eseguito con successo il caso d'uso \emph{UCC-1.2.3}
	\item \textbf{Post-condizioni:} la Standalone Application ha eseguito il comportamento associato ad un determinato evento
	\item \textbf{Flusso principale degli eventi:}
		\begin{enumerate}
			\item la Standalone Application esprime interesse relativamente al verificarsi di un evento (si guardi caso d'uso \emph{UCC-1.2.3})
			\item la Standalone Application osserva la notifica del verificarsi di un evento
			\item la Standalone Application analizza i parametri relativi all'evento
			\item la Standalone Application esegue il comportamento associato all'evento con i parametri forniti
		\end{enumerate}
\end{itemize}





\pagebreak

\subsubsection{UCR-1: Inizializza sessione}

%UCR-1: Inizializza sessione

\begin{figure}
\centering
\includegraphics[width=1.1\textwidth]{Immagini/Capitolo2/UseCases/UCR-generale.png}
\caption{Diagramma dei casi d'uso UCR-1, UCR-2, UCR-3}\label{fig:uc-ucr-generale}
\end{figure}

Nel proseguo dei paragrafi relativi ai casi d'uso \emph{UCR}, il nome dell'attore \emph{Remote Application} verrà abbreviato con la sigla \emph{RA}.

\begin{itemize}
	\item \textbf{Attori:} Remote Application (\emph{RA})
	\item \textbf{Scopo e descrizione:} la RA deve avere la possibilità di inizializzare una sessione persistente con il software server che memorizzi le informazioni i dati che devono essere conservati fra richieste differenti
	\item \textbf{Pre-condizioni:} la RA ha conoscenza dei parametri di connessione relativi ad una istanza del software server
	\item \textbf{Post-condizioni:} la RA ottiene un codice univoco che identifica la sessione creata
	\item \textbf{Flusso principale degli eventi:}
		\begin{enumerate}
			\item la RA esegue una connessione con il server (si guardi il caso d'uso \emph{UCR-1.1})
			\item la RA richiede l'inizializzazione di una sessione
				\begin{itemize}
					\item la RA può fornire al server i parametri di connessione di uno stream (si guardi il caso d'uso \emph{UCR-1.2})
					\item la RA può fornire al server i parametri di connessione di un listener (si guardi il caso d'uso \emph{UCR-1.3})
				\end{itemize}
			\item la RA memorizza l'identificativo di sessione
		\end{enumerate}
	\item \textbf{Flusso alternativo:}
		\begin{enumerate}
			\setcounter{enumi}{1}
			\item se la RA non ottiene l'identificativo di sessione, lo scenario prosegue dal punto 1 del flusso principale
		\end{enumerate}
\end{itemize}


\paragraph{UCR-1.1: Crea connessione}

\begin{itemize}
	\item \textbf{Attori:} RA
	\item \textbf{Scopo e descrizione:} la RA deve avere la possibilità di stabilire una connessione con il server tramite protocollo XML-RPC
	\item \textbf{Pre-condizioni:} la RA ha conoscenza dei parametri di connessione relativi ad una istanza del software server
	\item \textbf{Post-condizioni:} la RA ha stabilito una connessione. Permette l'esecuzione dei casi d'uso \emph{UCR-1}, \emph{UCR-1.2}, \emph{UCR-1.3}, \emph{UCR-3}
	\item \textbf{Flusso principale degli eventi:}
		\begin{enumerate}
			\item la RA stabilisce una connessione seguendo le specifiche del protocollo XML-RPC
		\end{enumerate}
	\item \textbf{Flusso alternativo:}
		\begin{enumerate}
			\item se la connessione fallisce, la RA attiva un protocollo di gestione dell'errore (la definizione del protocollo è materia della RA)
		\end{enumerate}
\end{itemize}


\paragraph{UCR-1.2: Registra stream}

\begin{itemize}
	\item \textbf{Attori:} RA
	\item \textbf{Scopo e descrizione:} la RA deve essere in grado di registrare uno stream remoto utilizzabile dal software server 
	\item \textbf{Pre-condizioni:} la RA ha eseguito con successo il caso d'uso \emph{UCR-1.1}
	\item \textbf{Post-condizioni:} il software server può utilizzare lo stream remoto come sostituto di uno locale. Permette l'esecuzione dei casi d'uso \emph{UCC-1.2.1} e \emph{UCC-1.2.2} anche per l'attore RA
	\item \textbf{Flusso principale degli eventi:}
		\begin{enumerate}
			\item la RA fornisce al software server i parametri di connessione di uno stream remoto
			\item la RA indica:
				\begin{enumerate}
					\item indirizzo e porta di connessione dello stream
					\item nome dello stream remoto
					\item nome dello stream per il software server
					\item un codice di controllo
				\end{enumerate}
			\item la RA riceve un segnale di conferma all'indirizzo indicato usando i parametri forniti
			\item lo stream remoto è indicato come sostituto di uno stream locale
		\end{enumerate}
	\item \textbf{Flusso alternativo:}
		\begin{enumerate}
			\setcounter{enumi}{2}
			\item se la RA non ottiene un segnale di conferma dopo un tempo massimo stabilito, la RA attiva un protocollo di gestione dell'errore (la definizione del protocollo è materia della RA)
			\item lo stream remoto viene ignorato
		\end{enumerate}
\end{itemize}


\paragraph{UCR-1.3: Registra listener}

\begin{itemize}
	\item \textbf{Attori:} RA
	\item \textbf{Scopo e descrizione:} la RA deve essere in grado di registrare un listener remoto utilizzabile dal software server 
	\item \textbf{Pre-condizioni:} la RA ha eseguito con successo il caso d'uso \emph{UCR-1.1}
	\item \textbf{Post-condizioni:} il software server può utilizzare il listener remoto come sostituto di uno locale. Permette l'esecuzione del caso d'uso \emph{UCC-1.2.3} anche per l'attore RA
	\item \textbf{Flusso principale degli eventi:}
		\begin{enumerate}
			\item la RA fornisce al software server i parametri di connessione di un listener remoto
			\item la RA indica:
				\begin{enumerate}
					\item indirizzo e porta di connessione del listener
					\item nome del listener remoto
					\item nome del listener per il software server
					\item il nome degli eventi per i quali registrare il listener
					\item un codice di controllo
				\end{enumerate}
			\item la RA riceve un segnale di conferma all'indirizzo indicato usando i parametri forniti
			\item il listener remoto è indicato come sostituto di uno locale
		\end{enumerate}
	\item \textbf{Flusso alternativo:}
		\begin{enumerate}
			\setcounter{enumi}{2}
			\item se la RA non ottiene un segnale di conferma dopo un tempo massimo stabilito, la RA attiva un protocollo di gestione dell'errore (la definizione del protocollo è materia della RA)
			\item il listener remoto viene ignorato
		\end{enumerate}
\end{itemize}



\pagebreak

\subsubsection{UCR-2: Distrugge sessione}

%UCR-2: Distrugge sessione

Il diagramma relativo al caso d'uso \emph{UCR-2}, data la natura complessa dello stesso, è accorpato a quello del caso d'uso \emph{UCR-1} fornito in \figurename~\ref{fig:uc-ucr-generale}.

\begin{itemize}
	\item \textbf{Attori:} RA
	\item \textbf{Scopo e descrizione:} la RA deve avere la possibilità di terminare e distruggere una sessione precedentemente inizializzata
	\item \textbf{Pre-condizioni:} la RA ha eseguito il caso d'uso \emph{UCR-1}
	\item \textbf{Post-condizioni:} il codice univoco ottenuto dall'esecuzione del caso d'uso \emph{UCR-1} non è più valido
	\item \textbf{Flusso principale degli eventi:}
		\begin{enumerate}
			\item la RA notifica la volontà di terminare la sessione al software server
			\item la RA riceve notifica di avvenuta chiusura della sessione
			\item la RA cancella dalla propria memoria il codice univoco di sessione
		\end{enumerate}
	\item \textbf{Flusso alternativo:}
		\begin{enumerate}
			\setcounter{enumi}{1}
			\item se la RA non riceve notifica di avvenuta chiusura della sessione dopo un tempo stabilito, notifica un avviso non invasivo
			\item lo scenario prosegue dal punto 3 del flusso principale
		\end{enumerate}
\end{itemize}
\pagebreak

\subsubsection{UCR-3: Richiede servizio}

%UCR-3: Richiede servizio

Il diagramma relativo al caso d'uso \emph{UCR-3}, data la natura complessa dello stesso, è accorpato a quello del caso d'uso \emph{UCR-1} fornito in \figurename~\ref{fig:uc-ucr-generale}.

\begin{itemize}
	\item \textbf{Attori:} RA
	\item \textbf{Scopo e descrizione:} la RA deve avere la possibilità di richiedere un servizio al software server
	\item \textbf{Pre-condizioni:} la RA ha eseguito il caso d'uso \emph{UCR-1.1}. La RA ha eseguito il caso d'uso \emph{UCR-1} se il servizio richiesto richiede l'inizializzazione di una sessione
	\item \textbf{Post-condizioni:} la RA ha ottenuto il risultato dell'elaborazione effettuata da un servizio
	\item \textbf{Flusso principale degli eventi:}
		\begin{enumerate}
			\item la RA richiede un servizio al software server
				\begin{itemize}
					\item la RA indica il nome del servizio richiesto
					\item la RA indica i parametri di esecuzione del servizio
					\item la RA può indicare il codice univoco di una sessione precedentemente inizializzata, se necessario
				\end{itemize}
			\item la RA attende il completamento dell'elaborazione
		\end{enumerate}
	\item \textbf{Flusso alternativo \#1:}
		\begin{enumerate}
			\item se la RA richiede un servizio non disponibile, la RA riceve un messaggio d'errore
			\item la RA attiva un protocollo di gestione dell'errore (la definizione del protocollo è materia della RA)
		\end{enumerate}

	\item \textbf{Flusso alternativo \#2:}
		\begin{enumerate}
			\item se la RA richiede un servizio disponibile, ma il codice di sessione è scaduto
			\item la RA esegue il caso d'uso \emph{UCR-1}
			\item lo scenario prosegue dal punto 1 del flusso principale
		\end{enumerate}

		
	\item \textbf{Flusso alternativo \#3:}
		\begin{enumerate}
			\setcounter{enumi}{1}
			\item se la RA non riceve risposta entro un tempo limite, la RA attiva un protocollo di gestione dell'errore (la definizione del protocollo è materia della RA)
		\end{enumerate}
\end{itemize}
\pagebreak


%\section{Requisiti}

%\subsection{Generale}

